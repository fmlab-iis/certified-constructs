\documentclass{llncs}

\usepackage{amsmath}
\usepackage{amsfonts}
\usepackage{amssymb}
%\usepackage{listings}
\usepackage[usenames,dvipsnames]{color}
\usepackage{xspace}
\usepackage{multirow}
\pagestyle{plain}
%\usepackage{graphicx}
\usepackage{algorithm,algpseudocode}
\usepackage[T1]{fontenc}
\usepackage{times}
%\usepackage{setspace}
%\setstretch{0.94}
\renewcommand{\algorithmicrequire}{\textbf{Input:}}
\renewcommand{\algorithmicensure}{\textbf{Output:}}


%\usepackage{a4wide}


\newcommand{\todo}[1]{\textcolor{blue}{TODO: {#1}}}
\newcommand{\hide}[1]{}

\newcommand{\defn}{\triangleq}
\newcommand{\dslcode}[1]{\mbox{\textsf{#1}}}
\newcommand{\code}[1]{\mbox{#1}}
\newcommand{\uscore}[0]{{\char95}}
\newcommand{\cond}[1]{{(\!|}{#1}{|\!)}}
\newcommand{\valueof}[1]{[\![{#1}]\!]}
\newcommand{\bbfB}{\mathbb{B}}
\newcommand{\bbfN}{\mathbb{N}}
\newcommand{\bbfZ}{\mathbb{Z}}
\newcommand{\bbfF}{\mathbb{K}}
\newcommand{\bbfGF}{\mathbb{GF}}
\newcommand{\bbfG}{\mathbb{G}}
\newcommand{\bbfV}{\mathbb{V}}
\newcommand{\bff}{\mathit{ff}}
\newcommand{\btt}{\mathit{tt}}
\newcommand{\St}{\mathit{St}}
\newcommand{\Tr}{\mathit{Tr}}
\newcommand{\coq}{\textsc{Coq}\xspace}
\newcommand{\openssh}{\textsc{OpenSSH}\xspace}
\newcommand{\openssl}{\textsc{OpenSSL}\xspace}
\newcommand{\singular}{\textsc{Singular}\xspace}
\newcommand{\qhasm}{\textsc{Qhasm}\xspace}
\newcommand{\gfverif}{\texttt{gfverif}\xspace}
\newcommand{\gallina}{\textsc{Gallina}\xspace}
\newcommand{\boolector}{\textsc{Boolector}\xspace}
\newcommand{\Prog}{\mathit{Prog}}
\newcommand{\Pred}{\mathit{Pred}}
\newcommand{\Nat}{\mathit{Nat}}
\newcommand{\Int}{\mathit{Int}}
\newcommand{\Var}{\mathit{Var}}
\newcommand{\Expr}{\mathit{Expr}}
\newcommand{\Stmt}{\mathit{Stmt}}
\newcommand{\Spec}{\mathit{Spec}}
\newcommand{\vx}{\vec{x}}
\newcommand{\vv}{\vec{v}}
\newcommand{\vn}{\vec{n}}
\newcommand{\vd}{\vec{d}}
\newcommand{\goesto}[1]{\overset{#1}{\Longrightarrow}}
\newcommand{\myprime}{\varrho}
\newcommand{\hoaretriple}[3]{\cond{#1}~{#2}~\cond{#3}}
\newcommand{\concat}[2]{{[}{#1}, {#2}{]}}
\newcommand{\mymapsto}[2]{{[}{#1} \leftarrow {#2}{]}}
\newcommand{\Gzero}{0_{\bbfG}}
\newcommand{\Gplus}{+_{\bbfG}}
\newcommand{\Geq}{=_{\bbfG}}
\newcommand{\Fplus}{+_{\bbfF}}
\newcommand{\Fminus}{-_{\bbfF}}
\newcommand{\Ftimes}{\cdot_{\bbfF}}
\newcommand{\Fzero}{0_{\bbfF}}
\newcommand{\Funit}{1_{\bbfF}}
\newcommand{\Feq}{=_{\bbfF}}
\newcommand{\Fdiv}{\div_{\bbfF}}
\newcommand{\bvdsl}{\textsc{bvCryptoLine}\xspace}
\newcommand{\zdsl}{\textsc{zCryptoLine}\xspace}
\renewcommand{\mod}{\mathbin{\textmd{mod}}}
\newcommand{\bbAtomic}{\mathit{bAtomic}}
\newcommand{\bvAtom}{\mathit{bAtom}}
\newcommand{\BV}{\bbfV} % Set of bit-vectors
\newcommand{\bv}{\mathit{bv}} % a function from Z to BV
\newcommand{\bvStmt}{\mathit{bStmt}} % Set of BV statements
\newcommand{\bvSt}{\mathit{bSt}} % Set of BV states
\newcommand{\bvTr}{\mathit{bTr}}
\newcommand{\bvProg}{\mathit{bProg}}
\newcommand{\bvExpa}{\mathit{bExp_a}}
\newcommand{\bvPreda}{\mathit{bPred_a}}
\newcommand{\bvExpr}{\mathit{bExp_r}}
\newcommand{\bvPredr}{\mathit{bPred_r}}
\newcommand{\bvPred}{\mathit{bPred}}
\newcommand{\bvSpec}{\mathit{bSpec}}
\newcommand{\bvst}{\nu} % prefix of BV states
\newcommand{\bvop}{\bbfV} % subscript of BV operators
\newcommand{\hi}{\mathit{hi}}
\newcommand{\lo}{\mathit{lo}}
\newcommand{\unsigned}[1]{|#1|}
\newcommand{\bvvalueof}[1]{[\![{#1}]\!]_\bbfV}
\newcommand{\zvalueof}[1]{[\![{#1}]\!]_\bbfZ}
\newcommand{\bvspectozspec}{\textsc{bvSpecTozSpec}}
\newcommand{\zSt}{\mathit{zSt}}
\newcommand{\zTr}{\mathit{zTr}}
\newcommand{\zst}{\mu} % prefix of Z states
\newcommand{\zExpr}{\mathit{zExpr}}
\newcommand{\zStmt}{\mathit{zStmt}}
\newcommand{\zSpec}{\mathit{zSpec}}
\newcommand{\zProg}{\mathit{zProg}}
\newcommand{\zPred}{\mathit{zPred}}
\newcommand{\wordsize}{\mathit{w}} % the assumed wordsize
\newcommand{\safe}{\mathit{safe}}
\makeatletter
\newcommand*\andar{\rotatebox[origin=c]{90}{$\dabar@\dabar@$}}
\makeatother

\makeatletter
\let\OldStatex\Statex
\renewcommand{\Statex}[1][3]{%
  \setlength\@tempdima{\algorithmicindent}%
  \OldStatex\hskip\dimexpr#1\@tempdima\relax}
\newcommand{\StatexIndent}[1][3]{%
  \setlength\@tempdima{\algorithmicindent}%
  \Statex\hskip\dimexpr#1\@tempdima\relax}
\makeatother

\algnewcommand\algorithmicmatch{\textbf{match}}
\algnewcommand\algorithmicwith{\textbf{with}}
\algnewcommand\algorithmiccase{\textbf{case}}
% New "environments"
\algdef{SE}[MATCH]{Match}{EndMatch}[1]{\algorithmicmatch\ {#1}\ \algorithmicwith}{\algorithmicend\ \algorithmicmatch}%
\algdef{SE}[CASE]{Case}{EndCase}[1]{\algorithmiccase\ {#1}:}{\algorithmicend\ \algorithmiccase}%
\algtext*{EndMatch}%
\algtext*{EndCase}%
\algnewcommand{\IfThenElse}[3]{% \IfThenElse{<if>}{<then>}{<else>}
  \State \algorithmicif\ #1\ \algorithmicthen\ #2\ \algorithmicelse\ #3}


\usepackage{xcolor}
\definecolor{linkcolor}{rgb}{0.65,0,0}
\definecolor{citecolor}{rgb}{0,0.65,0}
\definecolor{urlcolor}{rgb}{0,0,0.65}
\usepackage[colorlinks=true, backref=page, linkcolor=linkcolor, urlcolor=urlcolor, citecolor=citecolor]{hyperref}

\newif\ifpublic
%\publictrue

\title{Certified Verification of Algebraic Properties on
Low- \mbox{Level Mathematical Constructs in Cryptographic Programs}}

\ifpublic
\author{
Ming-Hsien Tsai
\and
Bow-Yaw Wang
\and
Bo-Yin Yang
%
%\thanks{}
}
\institute
{
Institute of Information Science\\
Academia Sinica\\
128 Section 2 Academia Road, Taipei 115-29, Taiwan\\
\email{mhtsai208@gmail.com, bywang@iis.sinica.edu.tw, by@crypto.tw}
}
\else
\author{\vspace*{-1.2cm}}
\institute{\vspace*{-1.2cm}\ }
\fi

\begin{document}

\maketitle

\begin{abstract}
  Mathematical constructs are necessary for computation on the
  underlying algebraic structures of cryptosystems. They are often
  written in assembly languages and optimized manually for
  efficiency. We develop a certified technique to verify mathematical
  constructs in X25519, the default elliptic curve Diffie-Hellman key
  exchange protocol used in \openssh. Our technique translates an
  algebraic specification of mathematical constructs into an algebraic
  problem. The algebraic 
  problem in turn is solved by the computer algebra system \singular. 
  The proof assistant 
  \coq certifies the translation and solution to algebraic
  problems. We report our case studies on verifying
  arithmetic computation over a large finite field and
  the Montgomery Ladderstep, a crucial loop  in X25519.
\end{abstract}

\vspace{-.5em}
\section{Introduction}
\vspace{-.5em}
\label{section:introduction}

In order to ensure computer security offered by cryptography,
cryptosystems must be realized by cryptographic programs where mathematical
constructs are required to compute on the underlying algebraic
structures of modern cryptosystems.
Such mathematical constructs are frequently invoked in cryptographic
programs. They are often written in low-level assembly languages and
manually optimized for efficiency. 
% Cryptographic primitives in cryptosystems are but sequences of
% algebraic operations on mathematical structures. 
Security guarantees of cryptographic programs
thus depend heavily on the correctness of mathematical constructs.
Cryptosystems could be compromised should programming
errors in mathematical constructs be exploited by adversaries.
%It is therefore of utmost importance to ensure the correctness of
%mathematical constructs used in cryptographic programs. 
In order to build secure cryptosystems, we develop a certified
technique to verify low-level mathematical constructs used in the
security protocol X25519 automatically in this paper.

X25519 is an Elliptic Curve Diffie-Hellman (ECDH) key exchange
protocol; it is a high-performance cryptosystem designed to 
use the secure elliptic curve Curve25519. Curve25519 is an elliptic
curve offering 128 bits of security when used with ECDH. In addition
to allowing high-speed elliptic curve arithmetic, it is easier to
implement properly, not covered by any known patents, and moreover
less susceptible to implementation pitfalls such as weak 
random-number generators. Its parameters were also selected by
easily described mathematical principles.
% without resorting to any random numbers or seeds. 
These characteristics make Curve25519 a
preferred choice for those who are leery of curves which might have
intentionally inserted
backdoors, such as those standardized by the United States National
Institute of Standards and Technology (NIST). 
Indeed, Curve25519 is currently the
de facto alternative to the NIST P-256 curve. Consequently, X25519 has
a wide variety of applications including the default key exchange
protocol in \openssh since 2014.

Most of the computation in X25519, in trade parlance, is in a
``variable base point multiplication,'' and the centerpiece 
is the Montgomery Ladderstep. This is usually a
large constant-time assembly program performing the
finite-field arithmetic that implements the mathematics on Curve25519.
Should the implementation of Montgomery ladderstep be incorrect, so
would that of X25519. Obviously for all its virtues, X25519 would be
pointless if its implementation is incorrect. This may be even more
relevant in cryptography than most of engineering, because cryptography is
one of the few disciplines with the concept of an omnipresent
adversary, constantly looking for the smallest edge --- and hence
eager to trigger any unlikely event. Revising a cryptosystem
due to rare failures potentially leading to a cryptanalysis is not
unheard of~\cite{HNPPSSW:03:IDFSNE}.
Thus, it is important for security that we can show the computations
comprising the Montgomery Ladderstep or (even better) the X25519
protocol to be correct. 
% However, such verification is not easy due the size both of the
% numbers in play (255 bits and more) and of the program itself
% (10,000+ machine instructions).

Several obstacles need be overcome in the verification of mathematical
constructs in X25519. The key exchange protocol is based on a
group induced by Curve25519. The elliptic
curve is in turn defined over the Galois field $\bbfGF(2^{255} - 19)$. 
To compute on the elliptic curve group, arithmetic computation over
$\bbfGF(2^{255} - 19)$ needs to be correctly implemented. 
Particularly, 255-bit multiplications modulo
$2^{255} - 19$ must be verified. Worse, commodity computing devices do
not support 255-bit arithmetic computation directly. Arithmetic over
the Galois field needs to be implemented by sequences of 32- or 64-bit
instructions of the underlying architectures. One has to
verify that a sequence of 32- or 64-bit instructions indeed
computes, say, a 255-bit multiplication over the finite field. Yet this
is only a single step in the operation on the elliptic curve group.
In order to compute the group operation, another sequence of
arithmetic computation over $\bbfGF(2^{255} - 19)$ is
needed. Particularly, a crucial step,
the Montgomery Ladderstep, requires 18 arithmetic 
computations over $\bbfGF(2^{255} - 19)$. The entire Ladderstep must be
verified to ensure security guarantees offered by Curve25519.
% and hence the ECDH key exchange protocol X25519.

In this paper, we focus on algebraic properties on low-level
implementations of mathematical constructs in cryptographic programs.
Mathematical constructs by their nature perform computation on
underlying algebraic structures. We aim to verify whether they perform
intended algebraic computation correctly. To this end, we propose the
domain specific language \mydsl for low-level 
mathematical constructs. Algebraic pre- and post-conditions of
programs written in \mydsl are specified as Hoare
triples. Such an algebraic specification is converted to static single 
assignment form and then translated into an algebraic problem (called 
the modular polynomial equation entailment problem). We use the computer
algebra system \singular to solve the algebraic problem. 
Program fragments irrelevant to algebraic properties are also
removed by slicing to reduce the size of algebraic problems.
We use \coq to certify the
correctness of translations, as well as solutions to algebraic
problems computed by \singular.

We report case studies on verifying mathematical constructs used in
the X25519 ECDH key exchange protocol. For each arithmetic operations
(such as addition, multiplication, and square) over $\bbfGF(2^{255} - 19)$,
their low-level real-world implementations are converted to our domain
specific langauge \mydsl manually. We specify mathematical
constructs with their algebraic properties. Mathematical constructs
are then verified against their algebraic
specifications automatically with our technique. 
The real-world implementation of the Montgomery Ladderstep is 
verified similarly.  


We have the following contributions:
\begin{itemize}
\item We propose a domain specific language \mydsl for modeling low-level
  mathematical constructs used in cryptographic programs.
\item We give a certified verification condition generator from
  algebraic specifications of programs to the modular polynomial
  equation entailment problem.
\item We verify arithmetic computation over a finite field of order
  $2^{255} - 19$ and a
  critical program (the Montgomery Ladderstep) automatically.
\item To the best of our knowledge, our work is the first automatic
  and certified verification on real cryptographic programs with
  minimal human intervention.
\end{itemize}

This paper is organized as follows. After preliminaries
(Section~\ref{section:preliminaries}), our domain specific
language is described in Section~\ref{section:domain-specific-language}. 
Section~\ref{section:translation}
presents the translation to the algebraic
problem. A certified solver for the algebraic problem is discussed in
Section~\ref{section:solving-algebraic-equations}. 
Section~\ref{section:evaluation} contains experimental results. It is
followed by conclusions.

%%% Local Variables: 
%%% mode: latex
%%% eval: (TeX-PDF-mode 1)
%%% eval: (TeX-source-correlate-mode 1)
%%% TeX-master: "certified_vcg"
%%% End: 


\vspace{-.5em}
\section{Preliminaries}
\vspace{-.5em}
\label{section:preliminaries}

We write $\mathbb{B} = \{ \mathit{ff}, \mathit{tt} \}$ for the Boolean
domain. Let $\bbfN$ and $\bbfZ$ denote all natural numbers and all integers
respectively. We use $[n]$ to denote the set $\{ 0, 1, \ldots, n \}$
for $n \in \bbfN$.

A \emph{monoid} $\mathcal{M} = (M, \epsilon, \cdot)$ consists of a set
$M$ and an associative binary operator $\cdot$ on $M$ with the
\emph{identity} $\epsilon \in M$. That is, $\epsilon \cdot m = m \cdot
\epsilon = m$ for every $m \in M$.
A \emph{group} $\mathcal{G} = (G, 0, +)$ is an algebraic structure
where $(G, 0, +)$ is a monoid and there is a $-a \in G$ such that
$(-a) + a = a + (-a) = 0$ for every $a \in G$. The element $-a$ is
called the \emph{inverse} of $a$. $\mathcal{G}$ is \emph{Abelian} if
the operator $+$ is commutative.
A \emph{ring} $\mathcal{R} = (R, 0, 1, +, \times)$ with $0 \neq 1$ is
an algebraic structure such that
\begin{itemize}
\item $(R, 0, +)$ is an Abelian group; 
\item $(R, 1, \times)$ is a monoid; and 
\item $\times$ is distributive over $+$: $a \times (b + c) = a \times
  b + a \times c$ for every $a, b, c \in R$.
\end{itemize}
If $\times$ is commutative, $\mathcal{R}$ is a \emph{commutative}
ring. 
% An \emph{integral domain} is a commutative ring where $a \times
% b = 0$ implies $a = 0$ or $b = 0$. 
A \emph{field} $\mathcal{F}$ $=$ $(F,$
$0,$ $1,$ $+,$ $\times)$ is a commutative ring where $(F\!\setminus\!\{0\}, 1, \times)$ is also
a group. $(\bbfN, 1, \times)$ is a monoid. $(\bbfZ, 0, 1, +, \times)$ 
is a commutative ring but not a field. 
For any prime number $\varrho$, the set $\{ 0, \ldots, \varrho - 1 \}$
with the addition and multiplication modulo $\varrho$ forms a \emph{Galois
field} of order $\varrho$ (written $\bbfGF(\varrho)$).
We focus on Galois fields of very large orders, in particular, $\myprime =
2^{255} - 19$.

Fix a set of variables $\vx$. $\mathcal{R}[\vx]$ is the set of
polynomials over $\vx$ with coefficients in the ring
$\mathcal{R}$. $\mathcal{R}[\vx]$ is a ring. A set $I \subseteq
\mathcal{R}[\vx]$ is an \emph{ideal} if 
\begin{itemize}
\item $f + g \in I$ for every $f, g \in I$; and
\item $h \times f \in I$ for every $h \in
  \mathcal{R}[\vx]$ and $f \in I$. 
\end{itemize}
Given $G \subseteq \mathcal{R}[\vx]$, $\langle G \rangle$ is the
minimal ideal containing $G$; $G$ are the \emph{generators}
of $\langle G \rangle$. The \emph{ideal membership}
problem is to decide if $f \in I$ for a given ideal $I$ and $f
\in \mathcal{R}[\vx]$.

Let $\BV^w$ be the set of all bit-vectors with a bit-width $w$.
The unsigned value of $b \in \BV^w$ is denoted by $\unsigned{b}$.
For a natural number or an integer $n$, let $\bv^w(n)$ be the two's complement representation of $n$ in a bit-width $w$.
We use the following common operators for fixed-width bit-vectors: $\BV^w +_\bvop \BV^w : \BV^w$ for addition, $\BV^w -_\bvop \BV^w : \BV^w$ for subtraction, $\BV^w \times_\bvop \BV^w : \BV^w$ for multiplication, $\BV^{w_1} ._\bvop \BV^{w_2} : \BV^{w_1 + w_2}$ for concatenation, $\BV^w \#_\bvop n : \BV^{w+n}$ for zero extension, $\BV^w \ll_\bvop n : \BV^w$ for left-shifting, $\BV^w \gg_\bvop n : \BV^w$ for logical right-shifting, and $\BV^w[i,j] : \BV^{i - j + 1}$ with $0 \leq j \leq i < w$ for bits extraction.
We also assume comparison operators $<_\bvop$ and $\leq_\bvop$ between unsigned values of bit-vectors.

Given a bit-vector $b \in \BV^{2w}$, define $\hi_\bvop(b) \defn b[2w - 1, w]$ for the extraction of higher $w$ bits, and $\lo_\bvop(b) \defn b[w - 1, 0]$ for the extraction of lower $w$ bits.
For operations $\bullet \in \{+_\bvop, -_\bvop, \times_\bvop\}$, we define their extended version $\bullet^\#$ which performs the original operation after doubling the width of operands by zero extension.
In the extended operations, the width of operands is doubled only once.
For example, for $b_1, b_2, b_3 \in \BV^{w}$, we have $b_1$ $+_\bvop^\#$ $b_1$ $\defn$ $(b_1$ $\#_\bvop$ $w)$ $+_\bvop$ $(b_2$ $\#_\bvop$ $w)$ and $b_1$ $+_\bvop^{\#}$ $b_2$ $+_\bvop^{\#}$ $b_3$ $\defn$ $(b_1$ $\#_\bvop$ $w)$ $+_\bvop$ $(b_2$ $\#_\bvop$ $w)$ $+_\bvop$ $(b_3$ $\#_\bvop$ $w)$.

%Define six more operations which takes high bits or low bits after $\bullet^{\#}$ operation.
%\[
%\begin{array}{rcl}
%b_1 \bullet^{\hi} b_2 & \defn & \hi_\bvop(b_1 \bullet^{\#} b_2) \\
%b_1 \bullet^{\lo} b_2 & \defn & \lo_\bvop(b_1 \bullet^{\#} b_2) \\
%\end{array}
%\]


%%% Local Variables:
%%% mode: latex
%%% eval: (TeX-PDF-mode 1)
%%% eval: (TeX-source-correlate-mode 1)
%%% TeX-master: "certified_vcg"
%%% End:


\vspace{-.5em}
\section{Domain Specific Language -- \mydsl}
\vspace{-.5em}
\label{section:domain-specific-language}

One of the big issues with modern cryptography is how the assumptions
match up with reality. In many situations, unexpected channels
through which information can leak to the attacker can cause the
cryptosystem to be broken. Typically this is about timing, or
electric power used. Implementations which are not defeated through
such means are called ``side-channel resilient'' implementations.
Clearly in such implementations the execution time are kept constant
(as much as possible).

This turns out in some cases to be harder than one would imagine.
This is chiefly because modern processors have caches and out of order
execution and multitasking. This makes it possible for one execution
thread, even when no privilege is conferred ,to affect the running
time of another -- simply by caching a sufficient amount of its own
data in correct locations through repeated accesses, and then
observing the running time of the other thread. The instructions in
the other thread which uses the ``evicted'' data (to make room for the
data of the eavesdropping thread) then has to take more time getting
their data back to the cache. Such attacks are quite practical.
\todo{insert reference in bib for D.J. Bernstein ``Cache Timing Attacks on AES'',
https://cr.yp.to/antiforgery/cachetiming-20050414.pdf}

Thus, the innocuous actions of (a) executing the CPU's conditional
branch instruction dependent on a secret bit, and (b) executing an
indirect load instruction using a secret value in the register as the
address, are both potentially dangerous leaks of information.

As such, modern cryptographic implementations often have to resort to
seemingly contrived sequences of conditional moves and arithmetic
manipulations to achieve the same result as simply using the CPU
instructions designed for the purpose. This of course slows the
computation down and waste resources but is inevitable because
secret-dependent conditional actions and table lookups are actually
very useful tasks that sometimes simply must be performed.

There are some side results of the above, one is that we are not often
faced with secret-dependent branching or table-lookups in the assembly
instructions, but a language describing cryptographic code might
include pseudo-instructions to cover instruction sequences, phrases in
the language if you will, that is used to achieve the same effect.

Consider the following syntactic class for expressions in our domain
specific language:

\begin{eqnarray*}
  N & ::= & \dslcode{1}\ |\ \dslcode{2}\ |\ \cdots\\
  Z & ::= & \cdots \ |\ \dslcode{-2}\ |\ \dslcode{-1} \ |\ 0\ |\ 
            \dslcode{1}\ |\ \dslcode{2}\ |\ \cdots\\
  V & ::= & \dslcode{x} \ |\ \dslcode{y} \ |\ \dslcode{z} \ |\ \cdots\\
  E & ::= &  Z \ |\ V \ |\  \dslcode{-}E \ |\ E \dslcode{+} E 
             \ |\ E \dslcode{-} E
             \ |\ E \times E \ |\ \dslcode{Pow} (E, N)
\end{eqnarray*}

We allow exact integers as constants in the domain specific
language. Variables are thus integer variables. An expression can be a
constant, a variable, or a negative expression. Addition, subtraction,
and multiplication of expressions are available. The expression
$\dslcode{Pow}(e, n)$ denotes $e^n$ for any expression $e$ and positive
integer $n$. More formally, let $\St \defn V \rightarrow
\bbfZ$ and $\nu \in \St$ be a \emph{state}. That is,
a {state} $\nu$ is a mapping from variables to integers. Define the
semantic function $\valueof{e}_{\nu}$ for an expression $e$ as follows.
\[
\begin{array}{rclcrcl}
  \valueof{{i}}_{\nu} & \defn & i \textmd{  for ${i} \in Z$} 
  & \hspace{.1\textwidth} &
  \valueof{{v}}_{\nu} & \defn & \nu({v}) 
     \textmd{  for ${v} \in V$}\\
  \valueof{-e}_{\nu} & \defn & -_{\bbfZ}\valueof{e}_{\nu}
  &&
  \valueof{e_0 + e_1}_{\nu} & \defn & 
     \valueof{e_0}_{\nu} +_{\bbfZ} \valueof{e_1}_{\nu} \\
  \valueof{e_0 - e_1}_{\nu} & \defn & 
     \valueof{e_0}_{\nu} -_{\bbfZ} \valueof{e_1}_{\nu}
  &&
  \valueof{e_0 \times e_1}_{\nu} & \defn & 
     \valueof{e_0}_{\nu} \times_{\bbfZ} \valueof{e_1}_{\nu} \\
  \valueof{\dslcode{Pow}(e, n)}_{\nu} & \defn & 
     (\valueof{e}_{\nu})^{\valueof{n}_{\nu}}
\end{array}
\]

Note that there is no expression for quotients, bitwise logical
operations. Bitwise left shifting however can be modeled by
multiplying $\dslcode{Pow}(2, n)$. Although \mydsl models a (very) 
small subset of assembly, it suffices to encode crucial low-level
mathematical constructs in X25519.

\begin{eqnarray*}
  S & ::= & V \leftarrow E 
            \ |\  \concat{V}{V} \leftarrow \dslcode{Split} (E, N)\\
\end{eqnarray*}

For low-level mathematical constructs, only assignments are allowed.
The statement $v \leftarrow e$ assigns the value of $e$
to the variable $v$. The statement $\concat{v_h}{v_l} \leftarrow
\dslcode{Split}(e, n)$ splits the value of $e$ into two parts;
the lowest $n$ bits are stored in $v_l$ and the higher bits
are stored in $v_h$. Define
\begin{eqnarray*}
\nu[\vv \mapsto \vd](v) & \defn &
\left\{
   \begin{array}{ll}
     d_i & \textmd{if $v = v_i \in \vv$}\\
     \nu(v) & \textmd{otherwise}
   \end{array}
\right.
\end{eqnarray*}
Consider the relation $\Tr \subseteq \St \times S \times \St$ defined
as follows. 
\begin{eqnarray*}
  (\nu, v \leftarrow e, \nu') \in \Tr & \textmd{if} &
  \nu' = \nu[v \mapsto \valueof{e}_{\nu}]\\
  (\nu, \concat{v_h}{v_l} \leftarrow \dslcode{Split} (e, n), \nu') \in
  \Tr & \textmd{if} &
  \nu' = \nu[v_h, v_l \mapsto h, l]
\end{eqnarray*}
where
$l = \valueof{e}_{\nu} \mod 2^{\valueof{n}_{\nu}}$ and
$h = (\valueof{e}_{\nu} - l) \div 2^{\valueof{n}_{\nu}}$.
Hence $(\nu, s, \nu') \in \Tr$ denotes that the state $\nu$ transits to 
the state $\nu'$ after executing the statement $s$.

\begin{eqnarray*}
  P & ::= & \epsilon \ |\ S; P
\end{eqnarray*}

A program is a sequence of statements. We denote the empty program by
$\epsilon$. Observe that conditional branches are not allowed in
cryptographic programs to prevent timing attacks. The semantics of a
program is defined by the relation 
$\Tr^* \subseteq \St \times P \times \St$:
\begin{eqnarray*}
  (\nu, \epsilon, \nu) \in \Tr^*\\
  (\nu, s; p, \nu'') \in \Tr^* & \textmd{if} &
    \textmd{there is a } \nu' \textmd{ with }
    (\nu, s, \nu') \in \Tr^* \textmd{ and }
    (\nu, p, \nu'') \in \Tr^*.
\end{eqnarray*}
Let $\nu, \nu' \in \St$ and $p = s_1; s_2; \cdots; s_n \in
P$. $\nu'$ is a state after executing $p$ from $\nu$
(written $\nu \goesto{p} \nu'$) if there are $\nu_1 = \nu, \nu_2,
\ldots, \nu_n, \nu_{n+1} = \nu'$ with $(\nu_i, s_i, \nu_{i+1}) \in
\Tr^*$ for every $1 \leq i < n$.

For algebraic specifications, $\top$ denotes the Boolean value
$\mathit{tt}$. A \emph{predicate} $q \in Q$ is an arbitrary
conjunction of (modulo) equations over expressions are admissible. For
$\nu \in \St$, we write $\bbfZ \models q[\nu]$ if $q$
evaluates to $\btt$ after replacing each variable $v$ with
$\nu(v)$; $\nu$ is moreover called a \emph{$q$-state}.

\begin{eqnarray*}
  Q & ::= & \top
     \ |\   E = E
     \ |\   E \equiv E \mod N
     \ |\   Q \wedge Q\\
  H & ::= & \cond{Q} P \cond{Q}
\end{eqnarray*}

We follow Hoare's formalism in algebraic specifications of
mathematical constructs~\cite{H:69:ABCP}. In an algebraic
specification $\hoaretriple{q}{p}{q'}$, $q$ and $q'$ are the \emph{pre}- and
\emph{post-conditions} of $p$ respectively. Given $q, q' \in Q$ and $p
\in P$, $\models \hoaretriple{q}{p}{q'}$ if for every $\nu, \nu' \in
\St$, $\bbfZ \models q[\nu]$ and $\nu \goesto{p} \nu'$ imply
$\bbfZ \models q'[\nu']$. Less formally, $\models
\hoaretriple{q}{p}{q'}$ if executing $p$ from a $q$-state always
results in a $q'$-state.

\begin{figure}[ht]
  \centering
  \[
  \begin{array}{lclcl}
    \begin{array}{rcl}
    \textmd{1:} && \dslcode{r}_0 \leftarrow \dslcode{x}_0; \\
    \textmd{2:} && \dslcode{r}_1 \leftarrow \dslcode{x}_1; \\
    \textmd{3:} && \dslcode{r}_2 \leftarrow \dslcode{x}_2; \\
    \textmd{4:} && \dslcode{r}_3 \leftarrow \dslcode{x}_3; \\
    \textmd{5:} && \dslcode{r}_5 \leftarrow \dslcode{x}_4; \\
    \end{array}
    &\hspace{.05\textwidth}&
    \begin{array}{rcl}
    \textmd{6:} && 
      \dslcode{r}_0 \leftarrow \dslcode{r}_0 + \dslcode{4503599627370458}; \\
    \textmd{7:} &&
      \dslcode{r}_1 \leftarrow \dslcode{r}_1 + \dslcode{4503599627370494}; \\
    \textmd{8:} &&
      \dslcode{r}_2 \leftarrow \dslcode{r}_2 + \dslcode{4503599627370494}; \\
    \textmd{9:} &&
      \dslcode{r}_3 \leftarrow \dslcode{r}_3 + \dslcode{4503599627370494}; \\
    \textmd{10:} && 
      \dslcode{r}_4 \leftarrow \dslcode{r}_4 + \dslcode{4503599627370494};\\
    \end{array}
    &\hspace{.05\textwidth}&
    \begin{array}{rcl}
    \textmd{11:} && \dslcode{r}_0 \leftarrow \dslcode{r}_0 - \dslcode{y}_0; \\
    \textmd{12:} && \dslcode{r}_1 \leftarrow \dslcode{r}_1 - \dslcode{y}_1; \\
    \textmd{13:} && \dslcode{r}_2 \leftarrow \dslcode{r}_2 - \dslcode{y}_2; \\
    \textmd{14:} && \dslcode{r}_3 \leftarrow \dslcode{r}_3 - \dslcode{y}_3; \\
    \textmd{15:} && \dslcode{r}_4 \leftarrow \dslcode{r}_4 - \dslcode{y}_4;
    \end{array}
  \end{array}
  \]
  \caption{Subtraction \dslcode{sub}}
%$(\dslcode{x}_0, \dslcode{x}_1, \dslcode{x}_2, \dslcode{x}_3,
%\dslcode{x}_4, \dslcode{y}_0, \dslcode{y}_1, \dslcode{y}_2,
%\dslcode{y}_3, \dslcode{y}_4)$
  \label{figure:dsl:subtraction}
\end{figure}

Figure~\ref{figure:dsl:subtraction} gives a simple yet real
implementation of subtraction over $\bbfGF(\myprime)$. 
In the figure, a number in $\bbfGF(\varrho)$ 
is represented by five 51-bit unsigned integers. The variables
$\dslcode{x}_0, \dslcode{x}_1, \dslcode{x}_2, \dslcode{x}_3,
\dslcode{x}_4$ for instance represent 
$\mathit{radix51}(\dslcode{x}_4, \dslcode{x}_3, \dslcode{x}_2,
\dslcode{x}_1, \dslcode{x}_0) \defn
2^{51 \times 4} \dslcode{x}_4 + 2^{51 \times 3} \dslcode{x}_3 + 
2^{51 \times 2} \dslcode{x}_2 + 2^{51 \times 1} \dslcode{x}_1 + 
2^{51 \times 0} \dslcode{x}_0$. The result of
subtraction is stored in the variables $\dslcode{r}_0, \dslcode{r}_1,
\dslcode{r}_2, \dslcode{r}_3, \dslcode{r}_4$. 
Given $0 \leq \dslcode{x}_0,$ $\dslcode{x}_1,$ $\dslcode{x}_2,$
$\dslcode{x}_3,$ $\dslcode{x}_4,$ $\dslcode{y}_0,$ $\dslcode{y}_1,$
$\dslcode{y}_2,$ $\dslcode{y}_3,$ $\dslcode{y}_4 < 2^{51}$, 
the specification of the program is therefore
\[
\begin{array}{c}
\cond{\top}\\
\dslcode{sub}\\
%(\dslcode{x}_0, \dslcode{x}_1, \dslcode{x}_2, \dslcode{x}_3,
%\dslcode{x}_4, \dslcode{y}_0, \dslcode{y}_1, \dslcode{y}_2,
%\dslcode{y}_3, \dslcode{y}_4)\\
\cond{\mathit{radix}(\dslcode{x}_4, \dslcode{x}_3, \dslcode{x}_2,
\dslcode{x}_1, \dslcode{x}_0) -
\mathit{radix51}(\dslcode{y}_4, \dslcode{y}_3, \dslcode{y}_2,
\dslcode{y}_1, \dslcode{y}_0)
\equiv
\mathit{radix51}(\dslcode{r}_4, \dslcode{r}_3, \dslcode{r}_2,
\dslcode{r}_1, \dslcode{r}_0)
\mod \myprime
}.
\end{array}
\]

Note that the variables $\dslcode{r}_i$'s are added with constants
after they are initizlied with $\dslcode{x}_i$'s but before
$\dslcode{y}_i$'s are subtracted from them. It is not hard to see that
$2\myprime = \mathit{radix51} (4503599627370458,$ $4503599627370494,$
$4503599627370494,$ $4503599627370494,$ $4503599627370494)$
after tedious computation. Hence $\mathit{radix51}(\dslcode{r}_4,$
$\dslcode{r}_3,$ $\dslcode{r}_2,$ $\dslcode{r}_1,$ $\dslcode{r}_0)$ $=$
$\mathit{radix51}(\dslcode{x}_4,$ $\dslcode{x}_3,$ $\dslcode{x}_2,$
$\dslcode{x}_1,$ $\dslcode{x}_0)$ $+$ $2 \myprime $ $-$
$\mathit{radix51}(\dslcode{y}_4,$ $\dslcode{y}_3,$ $\dslcode{y}_2,$
$\dslcode{y}_1,$ $\dslcode{y}_0)$ $\equiv $
$\mathit{radix51}(\dslcode{x}_4,$ $\dslcode{x}_3,$ $\dslcode{x}_2,$
$\dslcode{x}_1,$ $\dslcode{x}_0)$ $-$
$\mathit{radix51}(\dslcode{y}_4,$ $\dslcode{y}_3,$ $\dslcode{y}_2,$
$\dslcode{y}_1,$ $\dslcode{y}_0)$ $\mod \myprime $. The program in
Figure~\ref{figure:dsl:subtraction} is correct. The reason for
adding constants is to prevent underflow. If the constants were not
added, the subtraction in lines~11 to 15 could give negative and hence
%
incorrect results. \emph{Please note that ranges are a complicated
  issue.  Both addition and subtraction (Fig.~\ref{figure:dsl:subtraction}) gives results that are correct
  but possibly \textbf{overflowing} ($>\rho$), which can and must be
  accounted for later.} Indeed, characteristics of large Galois fields
are
%
regularly exploited in cryptographic programs for correctness and
efficiency. Our domain specific language can easily model such
specialized programming techniques and is most suitable for low-level
cryptographic programs.

%%% Local Variables: 
%%% mode: latex
%%% eval: (TeX-PDF-mode 1)
%%% eval: (TeX-source-correlate-mode 1)
%%% TeX-master: "certified_vcg"
%%% End: 


\vspace{-.5em}
\section{Translation to Algebraic Problems}
\vspace{-.5em}
\label{section:translation}

Given $q_a, q_a' \in \bvPreda$, $q_r, q_r' \in \bvPredr$, and $p \in \bvProg$, we reduce the problem of checking $\models \hoaretriple{q_a \andar q_r}{p}{q_a' \andar q_r'}$ to (1) the entailment problem of modular
polynomial equations over integer variables proving $\models \hoaretriple{q_a}{p}{q_a'}$ via an intermediate language \zdsl, (2) a range problem $\models \hoaretriple{q_r}{p}{q_r'}$, and (3) a safety check of program $p$.
The reduction is carried out by the following three transformations:
\begin{enumerate}
%\item Program slicing. To improve efficiency, fragments of the program $p$
%  irrelevant to the post-condition $q'$ are removed by program
%  slicing (Section~\ref{subsection:translation:slicing})~\cite{W:81:PS}.
\item Static single assignments. The program is transformed
  into static single assignments. Variables in pre- and
  post-conditions are also renamed
  (Section~\ref{subsection:translation:static-single-assignment})~\cite{AWZ:88:DQVP}.
\item \zdsl. The algebraic specification $\hoaretriple{q_a}{p}{q_a'}$ in \bvdsl is transformed to a specification in \zdsl so that the validity of the specification in \zdsl implies the validity of $\hoaretriple{q_a}{p}{q_a'}$ in \bvdsl if the program $p$ is safe. (Section~\ref{subsection:domain-specific-language-zdsl}).
\item Modular polynomial equations. Validity of algebraic specifications in \zdsl
  is reduced to the entailment of modular polynomial equations
  (Section~\ref{subsection:translation:multivariant-polynomial-equations})~\cite{H:07:AENTP}.
\end{enumerate}

For each transformation, we give an algorithm and establish the
correctness of the algorithm in \coq~\cite{YC:2004:ITPPDC}.
Specifically, semantics for \zdsl and validity of specifications in \zdsl are formalized.
The correctness of transformations is then certified by the proof assistant \coq.
For static single assignments, we
construct machine-checkable proofs for the soundness and completeness
of the transformation. For modular polynomial equations, another
\coq-certified proof shows the soundness of the transformation
from the validity of the algebraic specification to the entailment of
modular polynomial equations. In the following subsections,
transformations and their correctness are elaborated in details.


\vspace{-.5em}
\subsection{Program Slicing}
\vspace{-.5em}
\label{subsection:translation:slicing}

Consider the problem of checking $\models \hoaretriple{q}{p}{q'}$ for
arbitrary $q, q' \in \Pred$ and $p \in \Prog$. Since $q'$ is arbitrary, the
program $p$ may contain fragments irrevelant to $q'$. Program slicing
is a simple yet effective technique to improve efficiency of
verification through simplifying programs~\cite{W:81:PS}.

Here we slice programs in a backward manner. We initialize the set
of \emph{cared} variables to be the variables appeared in the
post-condition $q'$. Starting from the last statement, we check if
it assigns any cared variables. If not, we skip the last statement and
move to the last but one statement. If it does, we keep the last
statement, update the cared variables, and examine the second-to-last
statement. This process repeats until all statements of the
program are processed.

Our program slicing algorithm requires several auxiliary functions. 
Algorithm~\ref{algorithm:variables-in-expressions} shows how to
compute all variables occurred in an expression by a structural
induction. If the given expression is an integer,
the empty set is returned; if it is a variable, the singleton with the
variable is returned. Variables in other expressions are computed
recursively. 

\begin{algorithm}
  \begin{algorithmic}[1]
    \Function{VarsInExpr}{$e$}
    \Match{$e$}
      \Case{$v$}
        \Return $\{ v \}$
      \EndCase
      \Case{$i$}
        \Return $\emptyset$
      \EndCase
      \Case{$-e'$}
        \Return \Call{VarsInExpr}{$e'$}
      \EndCase
      \Case{$e' + e''$}
        \Return \Call{VarsInExpr}{$e'$} $\cup$ \Call{VarsInExpr}{$e''$}
      \EndCase
      \Case{$e' - e''$}
        \Return \Call{VarsInExpr}{$e'$} $\cup$ \Call{VarsInExpr}{$e''$}
      \EndCase
      \Case{$e' \times e''$}
        \Return \Call{VarsInExpr}{$e'$} $\cup$ \Call{VarsInExpr}{$e''$}
      \EndCase
      \Case{$\dslcode{Pow}(e', \uscore)$}
        \Return \Call{VarsInExpr}{$e'$}
      \EndCase
    \EndMatch
    \EndFunction
  \end{algorithmic}
  \caption{Variables Occurred in Expressions}
  \label{algorithm:variables-in-expressions}
\end{algorithm}

\begin{algorithm}
\begin{algorithmic}[1]
  \Function{VarsInPred}{$q$}
  \Match{$q$}
    \Case{$\top$}
      \Return $\emptyset$
    \EndCase
    \Case{$e' = e''$}
      \Return \Call{VarsInExpr}{$e'$} $\cup$ \Call{VarsInExpr}{$e''$}
    \EndCase
    \Case{$e' \equiv e'' \mod \uscore$}
      \Return \Call{VarsInExpr}{$e'$} $\cup$ \Call{VarsInExpr}{$e''$}
    \EndCase
    \Case{$q' \wedge q''$}
      \Return \Call{VarsInPred}{$q'$} $\cup$ \Call{VarsInPred}{$q''$}
    \EndCase
  \EndMatch
  \EndFunction
\end{algorithmic}
\caption{Variables Occurred in Predicates}
\label{algorithm:variables-in-predicates}
\end{algorithm}

Similarly, Algorithm~\ref{algorithm:variables-in-predicates} computes
the variables appeared in a predicate. Using the \coq specification
langauge \gallina, both algorithms are specified in the proof
assistant easily. 

\begin{algorithm}
  \begin{algorithmic}[1]
    \Function{SliceStatement}{$\mathit{vars}$, $s$}
    \Match{$s$}
      \Case{$v \leftarrow e$}
        \IfThenElse{$v \in \mathit{vars}$}
          {\Return $\langle$\Call{VarsInExpr}{$e$} 
            $\cup$ $(\mathit{vars} \setminus \{ v \})$, $s \rangle$}
          {\Return $\mathit{vars}$}
      \EndCase
      \Case{$\concat{v_h}{v_l} \leftarrow \dslcode{Split} (e, \uscore)$}
        \IfThenElse{$v_h$ or $v_l \in \mathit{vars}$}
          {\Return $\langle$\Call{VarsInExpr}{$e$} $\cup$ 
            $(\mathit{vars} \setminus \{ v_h, v_l \})$, $s \rangle$}
          {\Return $\mathit{vars}$}
      \EndCase
    \EndMatch
    \EndFunction
  \end{algorithmic}
  \caption{Slicing Statements}
  \label{algorithm:slicing-statements}
\end{algorithm}

For each statement, we check if the variables in the left hand side are
cared variables. If these variables are not, we leave the cared
variables intact. If they are cared variables, we update the cared
variables by excluding the variables on the left hand side but adding
the variables appeared on the right hand
side~(Algorithm~\ref{algorithm:slicing-statements}). In the algorithm, 
the parameter $\mathit{vars}$ denotes the set of cared variables.
If the given statement assigns to a cared variable, it is returned
with the updated cared variables by the algorithm. Otherwise, only
cared variables are returned.

To slice a program, our algorithm proceeds from the last statement
(Algorithm~\ref{algorithm:slicing-programs}). It invokes
\textsc{SliceStatement} to see if the current statement is relevant
and update cared variables if necessary. The algorithm recurses with
the updated cared variables and all but the last statements.

\begin{algorithm}
  \begin{algorithmic}[1]
    \Function{SliceProg}{$\mathit{vars}$, $p$}
    \Match{$p$}
      \Case{$\epsilon$}
        \Return $\epsilon$
      \EndCase
      \Case{$pp; s;$}
        \Match{\Call{SliceStatement}{$\mathit{vars}$, $s$}}
          \Case{$\mathit{vars}'$}
            \Return \Call{SliceProg}{$\mathit{vars}'$, $pp;$}
          \EndCase
          \Case{$\langle \mathit{vars}'$, $s' \rangle$}
            \Return \Call{SliceProg}{$\mathit{vars}'$, $pp$}$\ s';$
          \EndCase
        \EndMatch
      \EndCase
    \EndMatch
    \EndFunction
  \end{algorithmic}
  \caption{Slicing Programs}
  \label{algorithm:slicing-programs}
\end{algorithm}

Algorithms~\ref{algorithm:slicing-statements}
and~\ref{algorithm:slicing-programs} are specified in \gallina. Their
properties are 
also specified and proven formally in \coq. Particularly, the
soundness and completeness of our program slicing algorithm have been
certified in the proof assistant. Formally, we have the following theorem:

\begin{theorem}
  For every $q, q' \in \Pred$ and $p \in \Prog$,
  \begin{center}
  $\models \hoaretriple{q}{p}{q'}$ if and only if
  $\models \hoaretriple{q}{\textsc{SliceProg}(\textsc{VarsInPred}(q'), p)}
  {q'}$.
  \end{center}
  \label{theorem:program-slicing}
\end{theorem}


\vspace{-.5em}
\subsection{Static Single Assignments}
\vspace{-.5em}
\label{subsection:translation:static-single-assignment}

A program is in \emph{static single assignment} form if every variable
is assigned at most once~\cite{AWZ:88:DQVP}.
Our next task is to transform any algebraic specification
$\hoaretriple{q}{p}{q'}$ to an algebraic specification of $p$ in
static single  assignment form for any $q, q' \in \Pred$ and $p \in
\Prog$. Our transformation 
maintains a finite mapping $\theta$ from variables to
non-negative integers. For any variable $v$, $v^{\theta(v)}$ is
the most recently assigned copy of $v$. Only the most recent copies of
variables are referred in
expressions. Algorithm~\ref{algorithm:ssa-expressions} 
transforms expressions with the finite mapping $\theta$ by
structural induction. Integers are unchanged. For each variable, its
most recent copy is returned by looking up the mapping $\theta$. Other
expressions are transformed recursively. 

\begin{algorithm}
  \begin{algorithmic}[1]
    \Function{SSAExpr}{$\theta$, $e$}
    \Match{$e$}
      \Case{$i$} \Return $i$ \EndCase
      \Case{$v$} \Return $v^{\theta(e)}$ \EndCase
      \Case{$-e'$} \Return $-$\Call{SSAExpr}{$\theta$, $e'$} \EndCase
      \Case{$e' + e''$} 
        \Return \Call{SSAExpr}{$\theta$, $e'$} $+$ 
                \Call{SSAExpr}{$\theta$, $e''$}
      \EndCase
      \Case{$e' - e''$} 
        \Return \Call{SSAExpr}{$\theta$, $e'$} $-$ 
                \Call{SSAExpr}{$\theta$, $e''$}
      \EndCase
      \Case{$e' \times e''$} 
        \Return \Call{SSAExpr}{$\theta$, $e'$} $\times$ 
                \Call{SSAExpr}{$\theta$, $e''$}
      \EndCase
      \Case{$\dslcode{Pow}$($e'$, $n$)}
        \Return $\dslcode{Pow}$(\Call{SSAExpr}{$\theta$, $e'$}, $n$)
      \EndCase
    \EndMatch
    \EndFunction
  \end{algorithmic}
  \caption{Static Single Assignment Transformation for Expressions}
  \label{algorithm:ssa-expressions}
\end{algorithm}

Similarly, predicates must refer to most recent copies of 
variables. They are transformed according to the finite mapping
$\theta$. Thanks to the formalization of finite mappings in \coq. Both 
algorithms are easily specified in \gallina.

\hide{
\begin{algorithm}
  \begin{algorithmic}[1]
    \Function{SSAPred}{$\theta$, $q$}
    \Match{$q$}
      \Case{$\top$} \Return $\top$ \EndCase
      \Case{$e' = e''$} 
        \Return \Call{SSAExpr}{$\theta$, $e$} = \Call{SSAExpr}{$\theta$, $e'$}
      \EndCase
      \Case{$e' \equiv e'' \mod n$} 
        \Return \Call{SSAExpr}{$\theta$, $e$} $\equiv$ 
                \Call{SSAExpr}{$\theta$, $e'$} $\mod n$
      \EndCase
      \Case{$q' \wedge q''$}
        \Return \Call{SSAPred}{$\theta$, $q'$} $\wedge$
                \Call{SSAPred}{$\theta$, $q''$}
      \EndCase
    \EndMatch
    \EndFunction
  \end{algorithmic}
  \caption{Static Single Assignment Transformation for Predicates}
  \label{algorithm:ssa-predicates}
\end{algorithm}
}

Statement transformation is slightly more complicated
(Algorithm~\ref{algorithm:ssa-statements}). For 
expressions on the right hand side, they are transformed by the given
finite mapping $\theta$. The algorithm then updates $\theta$ and
replaces assigned variables with their latest copies. 
%In the algorithm, $\theta\mymapsto{v}{n}$ denotes updating the finite
%mapping $\theta$ by abusing the notation. 

\begin{algorithm}
  \begin{algorithmic}[1]
    \Function{SSAStmt}{$\theta$, $s$}
    \Match{$s$}
      \Case{$v \leftarrow e$}
        \State{$\theta' \leftarrow \theta\mymapsto{v}{\theta(v) + 1}$}
        \State{\Return $\langle \theta'$, 
                $v^{\theta'(v)} \leftarrow$ \Call{SSAExpr}{$\theta$, $e$}$\rangle$}
      \EndCase
      \Case{$\concat{v_h}{v_l} \leftarrow \dslcode{Split}(e, n)$}
        \State{$\theta_h \leftarrow \theta\mymapsto{v_h}{\theta(v_h) + 1}$}
        \State{$\theta_l \leftarrow \theta_h\mymapsto{v_l}{\theta_h(v_l) + 1}$}
        \State{\Return $\langle \theta_l$,
                $\concat{v_h^{\theta_h(v_h)}}{v_l^{\theta_l(v_l)}} \leftarrow
                \dslcode{Split}($\Call{SSAExpr}{$\theta$, $e$}$, n)
                \rangle$
              }
      \EndCase
    \EndMatch
    \EndFunction
  \end{algorithmic}
  \caption{Static Single Assignment Transformation for Statements}
  \label{algorithm:ssa-statements}
\end{algorithm}

It is straightforward to transform programs to static single
assignment form (Algorithm~\ref{algorithm:ssa-programs}). Using the
initial mapping from variables to $0$, 
the algorithm starts from the first statement and obtains an
updated mapping with the statement in static single assignment form. It
continues to transform the next statement with the updated mapping. 

\begin{algorithm}
  \begin{algorithmic}[1]
    \Function{SSAProg}{$\theta$, $p$}
    \Match{$p$}
      \Case{$\epsilon$}
        \Return $\langle \theta,$ $\epsilon \rangle$
      \EndCase
      \Case{$s; pp$}
        \State{$\langle \theta'$, $s' \rangle$ $\leftarrow$ 
                 \Call{SSAStmt}{$\theta$, $s$}}
        \State{$\langle \theta''$, $pp'' \rangle$ $\leftarrow$ 
                 \Call{SSAProg}{$\theta'$, $pp$}}
        \State{\Return $\langle \theta''$, $s'; pp'' \rangle$}
      \EndCase
    \EndMatch
    \EndFunction
  \end{algorithmic}
  \caption{Static Single Assignment for Programs}
  \label{algorithm:ssa-programs}
\end{algorithm}

Using the specifications of Algorithm~\ref{algorithm:ssa-statements}
and~\ref{algorithm:ssa-programs} in \gallina, properties of these
algorithms are formally proven in \coq. We first show that
Algorithm~\ref{algorithm:ssa-programs} indeed 
produces a program in static single assignment form.
\begin{lemma}
  Let $\theta_0(v) = 0$ for every $v \in \Var$, $p \in \Prog$, and
  $\langle \hat{\theta}, \hat{p} \rangle$ $=
  \textsc{SSAProg}(\theta_0, p)$. Then
  $\hat{p}$ is in static single assignment form.
  \label{lemma:ssa-programs}
\end{lemma}

To avoid ambiguity, we consider \emph{well-formed} programs where
\begin{itemize}
\item for every statement $\concat{v_h}{v_l} \leftarrow
  \dslcode{Split}(e, n)$ in the program, $v_h \neq v_l$; and
\item every variable in the program must be assigned to a value
  before being used. 
\end{itemize}
Well-formedness is preserved by our transformation. Formally, we have
\begin{lemma}
  Let $\theta_0(v) = 0$ for every $v \in \Var$ and $p \in \Prog$ a
  well-formed program. If $\langle \hat{\theta}, \hat{p} \rangle$ $=$ 
  $\textsc{SSAProg}(\theta_0, p)$, then $\hat{p}$ is well-formed.
  \label{lemma:ssa-well-formed}
\end{lemma}

The next theorem shows that our transformation is both sound and
complete. That is, an algebraic specification is valid if and only if its
corresponding specification in static single assignment form is valid.
\begin{theorem}
  Let $\theta_0(v) = 0$ for every $v \in \Var$. For every $q, q' \in \Pred$
  and $p \in \Prog$,
  \begin{center}
    $\models \hoaretriple{q}{p}{q'}$ if and only if
    $\models \hoaretriple{\textsc{SSAPred}(\theta_0, q)}
    {\hat{p}}
    {\textsc{SSAPred}(\hat{\theta}, q')}$
  \end{center}
  where $\langle \hat{\theta}, \hat{p} \rangle =
  \textsc{SSAProg}(\theta_0, p)$.
  \label{theorem:ssa}
\end{theorem}

\begin{figure}
  \centering
  \[
  \begin{array}{lclcl}
    \begin{array}{rcl}
    \textmd{1:} && {r}^0_0 \leftarrow {x}^0_0; \\
    \textmd{2:} && {r}^0_1 \leftarrow {x}^0_1; \\
    \textmd{3:} && {r}^0_2 \leftarrow {x}^0_2; \\
    \textmd{4:} && {r}^0_3 \leftarrow {x}^0_3; \\
    \textmd{5:} && {r}^0_4 \leftarrow {x}^0_4; \\
    \end{array}
    &\hspace{.05\textwidth}&
    \begin{array}{rcl}
    \textmd{6:} && 
      {r}^1_0 \leftarrow {r}^0_0 + {4503599627370458}; \\
    \textmd{7:} &&
      {r}^1_1 \leftarrow {r}^0_1 + {4503599627370494}; \\
    \textmd{8:} &&
      {r}^1_2 \leftarrow {r}^0_2 + {4503599627370494}; \\
    \textmd{9:} &&
      {r}^1_3 \leftarrow {r}^0_3 + {4503599627370494}; \\
    \textmd{10:} && 
      {r}^1_4 \leftarrow {r}^0_4 + {4503599627370494};\\
    \end{array}
    &\hspace{.05\textwidth}&
    \begin{array}{rcl}
    \textmd{11:} && {r}^2_0 \leftarrow {r}^1_0 - {y}^0_0; \\
    \textmd{12:} && {r}^2_1 \leftarrow {r}^1_1 - {y}^0_1; \\
    \textmd{13:} && {r}^2_2 \leftarrow {r}^1_2 - {y}^0_2; \\
    \textmd{14:} && {r}^2_3 \leftarrow {r}^1_3 - {y}^0_3; \\
    \textmd{15:} && {r}^2_4 \leftarrow {r}^1_4 - {y}^0_4;
    \end{array}
  \end{array}
  \]
  \caption{Subtraction \dslcode{subSSA} in Static Single Assignment Form}
% $(\dslcode{x}_0, \dslcode{x}_1, \dslcode{x}_2, \dslcode{x}_3,
% \dslcode{x}_4, \dslcode{y}_0, \dslcode{y}_1, \dslcode{y}_2,
% \dslcode{y}_3, \dslcode{y}_4)$
  \label{figure:translation:subtraction-ssa}
\end{figure}

\noindent
\emph{Example.}
Figure~\ref{figure:translation:subtraction-ssa} gives the subtraction program
\textsc{sub} in static single assignment form. Starting from $0$, the
index of a variable is incremented when the variable is assigned to an
expression. After the static single assignment translation, the
variables ${x}_i$'s, ${y}_i$'s are indexed by $0$ and
${r}_i$'s are indexed by $2$ for $0 \leq i \leq 4$. 
Subsequently, variables in pre- and post-conditions of the algebraic
specification for subtraction needs to be indexed. The
corresponding algebraic specification is as follows.
\[
\begin{array}{rcl}
\cond{\top} &
\dslcode{subSSA} &
\cond{
   \begin{array}{c}
     \mathit{radix51}({x}^0_4, {x}^0_3, {x}^0_2, {x}^0_1, {x}^0_0) -
     \mathit{radix51}({y}^0_4, {y}^0_3, {y}^0_2, {y}^0_1, {y}^0_0) \\
     \equiv
     \mathit{radix51}({r}^2_4, {r}^2_3, {r}^2_2, {r}^2_1, {r}^2_0)
     \mod \myprime
   \end{array}
}.
\end{array}
\]




\vspace{-.5em}
\subsection{Modular Polynomial Equation Entailment}
\vspace{-.5em}
\label{subsection:translation:multivariant-polynomial-equations}

The last step transforms any algebraic program specification to the 
modular polynomial equation entailment problem defined as follows. For
$q \in \Pred$, we write $q(\vx)$ to signify the free variables $\vx$
occurred in $q$. Given $q(\vx), q'(\vx) \in \Pred$, the \emph{modular
  polynomial equation entailment} problem decides whether 
$q(\vx) \implies q'(\vx)$ holds when $\vx$ are substituted for
arbitrary integers. That is, we want to check if $q(\vx) \implies
q'(\vx)$ evaluates to $\btt$ after each variable $x$ is replaced by
$\nu(x)$ for every valuation $\nu \in \St$. We write $\bbfZ \models
\forall \vx. q(\vx) \implies q'(\vx)$ if it is indeed the case.

Programs in static single assignment form can easily be transformed to
conjunctions of polynomial equations. We first show how to transform
statements into polynomial equations. An assignment statement is
translated to a polynomial equation with a single variable on the left
hand side. For a $\dslcode{Split}$ statement, it is transformed to an
equation with a linear expression of the assigned variables on the
left hand side (Algorithm~\ref{algorithm:polynomial-statements}). 
\begin{algorithm}
  \begin{algorithmic}[1]
    \Function{StmtToPolyEQ}{$s$}
    \Match{$s$}
      \Case{$v \leftarrow e$}
        \Return $v = e$
      \EndCase
      \Case{$\concat{v_h}{v_l} \leftarrow \dslcode{Split}(e, n)$}
        \Return $v_l + 2^n v_h = e$
      \EndCase
    \EndMatch
    \EndFunction
  \end{algorithmic}
  \caption{Polynomial Equation Transformation for Statements}
  \label{algorithm:polynomial-statements}
\end{algorithm}

A program in static single assignment form is transformed to the
conjunction of polynomial equations corresponding to its statements
(Algorithm~\ref{algorithm:polynomial-programs}). 

\begin{algorithm}
  \begin{algorithmic}[1]
    \Function{ProgToPolyEQ}{$p$}
    \Match{$p$}
      \Case{$\epsilon$} \Return $\top$ \EndCase
      \Case{$s; pp$}
        \Return \Call{StmtToPolyEQ}{$s$} $\wedge$
                \Call{ProgToPolyEQ}{$pp$}
      \EndCase
    \EndMatch
    \EndFunction
  \end{algorithmic}
  \caption{Polynomial Equation Transformation for Programs}
  \label{algorithm:polynomial-programs}
\end{algorithm}

Algorithm~\ref{algorithm:polynomial-statements}
and~\ref{algorithm:polynomial-programs} are specified straightforwardly 
in \gallina. We use the proof assistant \coq to prove properties
about the algorithms. First, note that $\textsc{ProgToPolyEQ}(p) \in
\Pred$ for every $p \in \Prog$. The following theorem shows that any
behavior of the program $p$ is a solution to the system of polynomial
equations $\textsc{ProgToPolyEQ}(p)$. In other words,
$\textsc{ProgToPolyEQ}(p)$ gives an abstraction of the program $p$.

\begin{theorem}
  Let $p \in \Prog$ be a well-formed program in static single assignment
  form. For every $\nu, \nu' \in \St$ with $\nu \goesto{p} \nu'$, 
  we have $\bbfZ \models \textsc{ProgToPolyEQ}(p) [\nu']$.
\end{theorem}

Definition~\ref{definition:program-entailment} gives the modular
polynomial equation entailment problem corresponding to an algebraic
program specification.
\begin{definition}
  For any $q, q' \in \Pred$ and $p \in \Prog$ in static single assignment
  form, define
  \begin{eqnarray*}
    \Pi(\hoaretriple{q}{p}{q'}) & \defn &
    q(\vx) \wedge \varphi(\vx) \implies q'(\vx)
  \end{eqnarray*}
  where $\varphi(\vx) =
  \textsc{ProgramToPolyEQ}(p)$. 
  \label{definition:program-entailment}
\end{definition}

\begin{figure}
  \centering
  \[
  \begin{array}{l}
  \top \wedge
  \left(
  \begin{array}{lclcl}
    \begin{array}{rclc}
      {r}^0_0 & = & {x}^0_0 & \wedge \\
      {r}^0_1 & = & {x}^0_1 & \wedge \\
      {r}^0_2 & = & {x}^0_2 & \wedge \\
      {r}^0_3 & = & {x}^0_3 & \wedge \\
      {r}^0_4 & = & {x}^0_4 & \wedge \\
    \end{array}
    &\hspace{.01\textwidth}&
    \begin{array}{rclc}
      {r}^1_0 & = & {r}^0_0 + {4503599627370458} & \wedge \\
      {r}^1_1 & = & {r}^0_1 + {4503599627370494} & \wedge \\
      {r}^1_2 & = & {r}^0_2 + {4503599627370494} & \wedge \\
      {r}^1_3 & = & {r}^0_3 + {4503599627370494} & \wedge \\
      {r}^1_4 & = & {r}^0_4 + {4503599627370494} & \wedge\\
    \end{array}
    &\hspace{.01\textwidth}&
    \begin{array}{rclc}
      {r}^2_0 & = & {r}^1_0 - {y}^0_0 & \wedge \\
      {r}^2_1 & = & {r}^1_1 - {y}^0_1 & \wedge \\
      {r}^2_2 & = & {r}^1_2 - {y}^0_2 & \wedge \\
      {r}^2_3 & = & {r}^1_3 - {y}^0_3 & \wedge \\
      {r}^2_4 & = & {r}^1_4 - {y}^0_4
    \end{array}
  \end{array}
  \right)\\
    \hspace{.05\textwidth}
    \implies 
    \mathit{radix}({x}^0_4, {x}^0_3, {x}^0_2, {x}^0_1, {x}^0_0) -
    \mathit{radix51}({y}^0_4, {y}^0_3, {y}^0_2, {y}^0_1, {y}^0_0)
    \equiv
    \mathit{radix51}({r}^2_4, {r}^2_3, {r}^2_2, {r}^2_1, {r}^2_0)
    \mod \myprime
  \end{array}
  \]
  \caption{Modular Polynomial Equation Entailment for \dslcode{subSSA}}
  \label{figure:translation:subtraction-polynomial}
\end{figure}

\noindent
\emph{Example}.
The modular polynomial equation entailment problem coresponding to the
algebraic specification of subtraction is shown in
Figure~\ref{figure:translation:subtraction-polynomial}. The problem
has 15 polynomial equality constraints with 25 integer variables. We would like
to see if $\mathit{radix51}({r}^2_4, {r}^2_3, {r}^2_2,
{r}^2_1, {r}^2_0)$ is the difference between $\mathit{radix}({x}^0_4,
{x}^0_3, {x}^0_2, {x}^0_1, {x}^0_0)$ and $\mathit{radix51}({y}^0_4,
{y}^0_3, {y}^0_2, {y}^0_1, {y}^0_0)$ in $\bbfGF(\varrho)$ under the
constraints. 

Theorem~\ref{theorem:program-to-q-soundness} establishes the
soundnesss of Algorithm~\ref{algorithm:polynomial-programs}. Its proof
is again certified by \coq.
\begin{theorem}
  \label{theorem:program-to-q-soundness}
  Let $q, q' \in \Pred$ be predicates, and $p \in \Prog$ a program in
  static single assignment form and well-formed. 
  If $\bbfZ \models \forall \vx.\Pi(\hoaretriple{q}{p}{q'})$, then
  $\models$ $\hoaretriple{q}{p}{q'}$.
\end{theorem}



\vspace{-.5em}
\subsubsection*{Summary of Translation}
\vspace{-.5em}

Consider any predicates $q, q' \in \Pred$ and well-formed program $p \in
\Prog$. Let $\theta_0 (v) = 0$ for every $v \in \Var$. By
Theorem~\ref{theorem:program-slicing}, \ref{theorem:ssa}, and
\ref{theorem:program-to-q-soundness}, we have the following deduction:
\begin{equation*}
  \begin{array}{cll}
    & \models \hoaretriple{q}{p}{q'}\\
    \Leftrightarrow
    & \models 
      \hoaretriple{q}{\textsc{SliceProg}(\textsc{VarsInPred}(q'), p)}{q'}
    & \textmd{ (Theorem~\ref{theorem:program-slicing})}\\
    \Leftrightarrow
    & \models
      \hoaretriple{\textsc{SSAPred}(\theta_0, q)}
      {\hat{p}}
      {\textsc{SSAPred}(\hat{\theta}, q')}
    & \textmd{ (Theorem~\ref{theorem:ssa})}\\
    &
      \textmd{where } \langle \hat{\theta}, \hat{p} \rangle = 
      \textsc{SSAProg} (\theta_0, \textsc{SliceProg}
      (\textsc{VarsInPred}(q'), p))\\
    \Leftarrow
    & \bbfZ \models \forall \vx.
      \Pi (\hoaretriple{\textsc{SSAPred}(\theta_0, q)} {\hat{p}}
      {\textsc{SSAPred}(\hat{\theta}, q')})
    & \textmd{ (Theorem~\ref{theorem:program-to-q-soundness})}
  \end{array}
\end{equation*}


\vspace{-.5em}
\section{Solving Modular Polynomial Equation Entailment Problem}
\vspace{-.5em}
\label{section:solving-algebraic-equations}

%By Theorem~\ref{theorem:program-to-q-soundness}, 
It remains to show 
\[
\begin{array}{l}
  \bbfZ \models \forall \vx. % \in \bbfZ^{|\vx|}.
  \bigwedge\limits_{i \in [I]} e_i (\vx) = e'_i (\vx) \wedge
  \bigwedge\limits_{j \in [J]} f_j (\vx) \equiv f'_j (\vx) \mod n_j
  \implies
  \\
  \hspace{.3\textwidth}
  \bigwedge\limits_{k \in [K]} g_k (\vx) = g'_k (\vx) \wedge
  \bigwedge\limits_{l \in [L]} h_l (\vx) \equiv h'_l (\vx) \mod m_l
\end{array}
\]
where
$e_i (\vx), e'_i (\vx), f_j (\vx), f'_j (\vx),
 g_k (\vx), g'_k (\vx), h_l (\vx), h'_l (\vx) \in
 \bbfZ[\vx]$, $n_j, m_l \in \bbfN$ for $i \in [I]$, $j \in [J]$, $k
 \in [K]$, and $l \in [L]$. Since the
consequence is a conjunction of (modular) equations, it suffices to 
prove one conjunct at a time. That is, we aim to show
\begin{equation*}
  \label{eq:polynomial-equation-implicant}
  \bbfZ \models \forall \vx. % \in \bbfZ^{|\vx|}.
  \bigwedge\limits_{i \in [I]} e_i (\vx) = e'_i (\vx) \wedge
  \bigwedge\limits_{j \in [J]} f_j (\vx) \equiv f'_j (\vx) \mod n_j
  \implies
  g (\vx) = g' (\vx); \textmd{ or}
\end{equation*}
 \begin{equation*}
   \label{eq:modular-polynomial-equation-implicant}
   \bbfZ \models \forall \vx. % \in \bbfZ^{|\vx|}.
   \bigwedge\limits_{i \in [I]} e_i (\vx) = e'_i (\vx) \wedge
   \bigwedge\limits_{j \in [J]} f_j (\vx) \equiv f'_j (\vx) \mod n_j
   \implies
   h (\vx) \equiv h' (\vx) \mod m
 \end{equation*}
where $e_i (\vx), e'_i (\vx), f_j (\vx), f'_j (\vx), g (\vx), g'
(\vx), h (\vx), h' (\vx)
\in \bbfZ[\vx]$ for $i \in [I], j \in [J]$, and $m \in \bbfN$. 

It is not hard to rewrite modular polynomial equations in antecedents
of the above implications. For instance, the first implication is
equivalent to 
\[
\bbfZ \models \forall \vx. % \in \bbfZ^{|\vx|}.
\bigwedge\limits_{i \in [I]} e_i (\vx) = e'_i (\vx) \wedge
\bigwedge\limits_{j \in [J]} [\exists d_j. f_j (\vx) = f'_j (\vx) + d_j \cdot n_j]
\implies
g (\vx) = g' (\vx),
\]
which in turn is equivalent to
\[
\bbfZ \models \forall \vx \forall \vd. % \in \bbfZ^{|\vx|} \vd \in \bbfZ^{J}.
\bigwedge\limits_{i \in [I]} e_i (\vx) = e'_i (\vx) \wedge
\bigwedge\limits_{j \in [J]} f_j (\vx) = f'_j (\vx) + d_j \cdot n_j
\implies
g (\vx) = g' (\vx).
\]

It hence suffices to consider the following
\emph{polynomial equation entailment} problem:
\begin{equation}
  \label{eq:reduced-polynomial-equation-implicant}
  \bbfZ \models \forall \vx. % \in \bbfZ^{|\vx|}.
  \bigwedge\limits_{i \in [I]} e_i (\vx) = e'_i (\vx)
  \implies
  g (\vx) = g' (\vx); \textmd{ or}
\end{equation}
 \begin{equation}
   \label{eq:reduced-modular-polynomial-equation-implicant}
   \bbfZ \models \forall \vx. % \in \bbfZ^{|\vx|}.
   \bigwedge\limits_{i \in [I]} e_i (\vx) = e'_i (\vx)
   \implies
   h (\vx) \equiv h' (\vx) \mod m
 \end{equation}
 where $e_i (\vx), e'_i (\vx), g (\vx), g' (\vx), h (\vx), h' (\vx)
 \in \bbfZ[\vx]$ for $i \in [I]$ and $m \in \bbfN$~\cite{H:07:AENTP}.

We solve the polynomial equation entailment
problems~(\ref{eq:reduced-polynomial-equation-implicant}) 
and~(\ref{eq:reduced-modular-polynomial-equation-implicant}) via
the ideal membership problem~\cite{H:07:AENTP,BS:16:GFEV}. 
For~(\ref{eq:reduced-polynomial-equation-implicant}), consider the
ideal $I = \langle e_i(\vx) - e'_i(\vx) \rangle_{i \in [I]}$. Suppose
$g(\vx) - g'(\vx) \in I$. Then there are $u_i(\vx) \in \bbfZ[\vx]$
(called \emph{coefficients}) such that 
\begin{equation}
  \label{eq:reduced-polynomial-equation-witnesses}
  g(\vx) - g'(\vx) = \sum\limits_{i \in [I]} u_i (\vx) [e_i (\vx) - e'_i (\vx)].
\end{equation}
Hence $g(\vx) - g'(\vx) = 0$ follows from  the polynomial equations
$e_i (\vx) = e'_i (\vx)$ for $i \in [I]$. Similarly, it
suffices to check if $h(\vx) - h'(\vx) \in \langle m, e_i(\vx) -
e'_i(\vx) \rangle_{i \in [I]}$
for~(\ref{eq:reduced-modular-polynomial-equation-implicant}).
If so, there are $u, u_i(\vx) \in \bbfZ[\vx]$ such that
\begin{equation}
  \label{eq:reduced-modular-polynomial-equation-witnesses}
  h(\vx) - h'(\vx) = u(\vx) \cdot m + \sum\limits_{i \in [I]} u_i (\vx)
  [e_i (\vx) - e'_i (\vx)]. 
\end{equation}
Thus $h(\vx) \equiv h'(\vx) \mod m$ as required.
The reduction to the ideal membership problem however is 
incomplete. Consider $\bbfZ \models \forall x. x^2 + x
\equiv 0 \mod 2$ but $x^2 + x \not\in \langle 2
\rangle$~\cite{H:07:AENTP}. 

Two \coq tactics are available to
find formal proofs for the polynomial equation entailment
problems~\cite{P:08:CGBP,P:10:CGBP}. 
The tactic \dslcode{nsatz} proves the entailment problem of
the form in~(\ref{eq:reduced-polynomial-equation-implicant}); the
tactic \dslcode{gbarith} is able to prove the form
in~(\ref{eq:reduced-modular-polynomial-equation-implicant}). 
The ideal membership problem can be solved by finding a Gr\"obner
basis for the ideal. 
Both tactics solve the polynomial equation entailment problem by
computing Gr\"obner bases for induced ideals. Finding Gr\"obner bases
for ideals however is NP-hard because it allows us to solve a
system of equations over the Boolean field~\cite{GJ:1979:CAI}. 
Low-level mathematical constructs can have
hundreds of polynomial equations 
in~(\ref{eq:reduced-polynomial-equation-implicant})
or~(\ref{eq:reduced-modular-polynomial-equation-implicant}). 
Both \coq tactics fail to solve such problems in a reasonable
amount of time. 

We develop two heuristics to solve the polynomial equation
entailment problem more effectively. Note that the polynomial equations
generated by Algorithm~\ref{algorithm:polynomial-programs} are of the  
forms: $x = e$ (from assignment statements) or $x + 2^c y
= e$ (from \dslcode{Split} statements). Such polynomial equations can
safely be removed after every occurrences of $x$ are replaced with $e$
or $e - 2^c y$ respectively. The number of generators of the induced
ideal is hence reduced. We define a \coq
tactic to simplify polynomial equation entailment problems by 
rewriting variables and then removing polynomial equations.

To further improve scalability, we use the computer algebra system
\singular to solve the ideal membership problem~\cite{GP:08:SICA}. 
Our tactic submits the membership problem to \singular and 
obtains coefficients from the computer algebra system. 
Since algorithms used in \singular might be implemented incorrectly, 
our \coq
tactic then certifies the coefficients by checking the
equation~(\ref{eq:reduced-polynomial-equation-witnesses})
or~(\ref{eq:reduced-modular-polynomial-equation-witnesses}) to ensure
the polynomial equation entailment problem is correctly
solved. Soundness of our technique therefore does not rely on the
external solver \singular.

%%% Local Variables: 
%%% mode: latex
%%% eval: (TeX-PDF-mode 1)
%%% eval: (TeX-source-correlate-mode 1)
%%% TeX-master: "certified_vcg"
%%% End: 


\vspace{-.5em}
\section{Evaluation}
\vspace{-.5em}
\label{section:evaluation}

We evaluate our techniques in real-world low-level mathematical
constructs in X25519. 
In elliptic curve cryptography, arithmetic computation over
large finite fields is required. For
instance, Curve25519 defined by $y^2 = x^3 + 486662 x^2 + x$ 
is over the Galois field $\bbfF = \bbfGF(\varrho)$ with
$\varrho = 2^{255} - 19$. To
make the field explicit, we rewrite its definition as:
\begin{equation}
  \label{eq:curve25519}
  y \Ftimes y \Feq x \Ftimes x \Ftimes x \Fplus
  486662 \Ftimes x \Ftimes x \Fplus x.
\end{equation}

Since arithmetic computation is over $\bbfF$ whose
elements can be represented by 255-bit numbers,
any pair $(x, y)$ satisfying (\ref{eq:curve25519}) (called a
\emph{point} on the curve) can be represented by a pair of 255-bit
numbers. It can be shown that points on Curve25519 with the point at
infinity as the unit (denoted $\Gzero$) form a commutative group $\bbfG = (G, \Gplus, \Gzero)$
with $G = \{ (x, y) : x, y \textmd{ satisfying } (\ref{eq:curve25519})
\}$. Let $P_0 = (x_0, y_0), P_1 = (x_1, y_1) \in G$. We have $-P_0 =
(x_0, -y_0)$ and $P_0 \Gplus P_1 = (x, y)$  where
\begin{eqnarray}
  m & = & (y_1 \Fminus y_0) \Fdiv (x_1 \Fminus x_0)\\\nonumber
  x & = & m \Ftimes m \Fminus 486662 \Fminus x_0 \Fminus x_1\\ \nonumber
  y & = & %m \Ftimes (x_0 \Fminus x) \Fminus y_0 
     %= m \Ftimes (x_1 \Fminus x) \Fminus y_1.
          (2 \Ftimes x_0 \Fplus x_1 \Fplus 486662 )\Ftimes m
          \Fminus m \Ftimes m \Ftimes m\Fminus y_0
  \label{eq:curve25519-group}
\end{eqnarray}
when $P_0 \neq \pm P_1$. Other cases ($P_0 = \pm P_1$) are defined
similarly~\cite{C:96:CCANT}.
$\bbfG$ and similar elliptic curve groups
are the main objects in elliptic curve cryptography. It is
essential to implement the commutative binary operation $\Gplus$ very
efficiently in practice.
\hide{
When $P_0=-P_1$, the sum is the point at infinity; When $P_0= P_1$,
$P_0 \Gplus P_1 = (x, y)$ where
\begin{eqnarray}
  \label{eq:curve25519-dbl}
  m &=& (3  \Ftimes x_1 \Ftimes x_1 \Fplus 2\Ftimes a\Ftimes x_1\Fplus 1) 
        \Fdiv (2\Ftimes b\Ftimes y_1) \\ \nonumber
   x &=& b \Ftimes m \Ftimes m \Fminus a \Fminus 2 \Ftimes x_1 \\ \nonumber
   y &=& (3 \Ftimes x1\Fplus a) \Ftimes m \Fminus b \Ftimes m \Ftimes m \Ftimes m\Fminus y_1
\end{eqnarray}
%
}

%%% Local Variables: 
%%% mode: latex
%%% eval: (TeX-PDF-mode 1)
%%% eval: (TeX-source-correlate-mode 1)
%%% TeX-master: "certified_vcg"
%%% End: 


\vspace{-.5em}
\subsection{Arithmetic Computation over $\bbfGF(2^{255}-19)$}
\vspace{-.5em}
\label{subection:evaluation:multiplication}
 
The operator $\Gplus$ is defined by arithmetic computation over
$\bbfF$. %In order to compute $P_0 \Gplus P_1$ for $P_0, P_1 \in G$,
Arithmetic operations over $\bbfF$ need to be implemented. Recall that
an element in $\bbfF$ is represented by a 256-bit number. 
% The na\"ive implementation for arithmetic operations over $\bbfF$
% would require rithmetic computation over $\bbfZ$. 
Arithmetic computation for 255-bit integers however is not yet
available in commodity computing devices as of the year
2017. Long-integer arithmetic has to be carried out by limbs where a
\emph{limb} is a 32- or 64-bit number depending on the underlying
computer architectures. Figure~\ref{figure:dsl:subtraction}
(Section~\ref{section:domain-specific-language}) is an
efficient and secure implementation of subtraction for the AMD64
architecture.  Notice how it is much more parallel, and also has
constant execution time.


\hide{
In practice, efficient long-integer arithmetic however is more
complicated. Consider subtracting a long integer from another. The
na\"ive implementation would simply subtract by limbs, propagate carry
flags, and add the prime number $\varrho$ if necessary. 
Executime time of the na\"ive subtraction however varies when minuend
is less than subtrahend. It thus allows timing attacks and is 
insecure. The na\"ive implementation of 255-bit subtraction should
never be used in cryptographic programs.
}

Multiplication is another interesting but much more
complicated operation. The na\"ive implementation for 255-bit
multiplication would compute a 510-bit result and then find the
corresponding 255-bit representation by division. 
An efficient implementation for 255-bit multiplication avoids the
division by performing modulo operations aggressively. Observe that
$2^{255} \Feq 19$ in $\bbfGF(\varrho)$. Whenever an intermediate
result of the form $c \Ftimes 2^{255}$ is obtained, it is replaced by
$c \Ftimes 19$. Indeed, the most efficient implementation for AMD64
architecture does not perform any division \todo{reference?}.

\hide{
Recall that 
elements in $\bbfF$ are represented by five limbs of 51-bit unsigned
integers. Consider multiplying two 255-bit values
\begin{equation*}
  \begin{array}{lcccccccccccl}
    x & = & \mathit{radix51} (x_4, x_3, x_2, x_1, x_0) & = &
            2^{51 \times 4} x_4 & + & 2^{51 \times 3} x_3 & + &
            2^{51 \times 2} x_2 & + & 2^{51 \times 1} x_1 & + &
            2^{51 \times 0} x_0 \textmd{ and}\\
    y & = & \mathit{radix51} (y_4, y_3, y_2, y_1, y_0) & = &
            2^{51 \times 4} y_4 & + & 2^{51 \times 3} y_3 & + &
            2^{51 \times 2} y_2 & + & 2^{51 \times 1} y_1 & + &
            2^{51 \times 0} y_0.
  \end{array}
\end{equation*}
The intermediate results $2^{255} x_4 y_1$, $2^{255} x_3 y_2$,
$2^{255} x_2 y_3$, and $2^{255} x_1 y_4$ can all be replaced by 
$19 x_4 y_1$, $19 x_3 y_2$, $19 x_2 y_3$, and $19 x_1 y_4$,
respectively. Division is not needed in such implementations.
}

\hide{
We describe but a couple of optimizations in efficient implementations
of arithmetic operations over $\bbfF$. Real-world implementations
are necessarily optimized with various algebraic properties in
$\bbfF$. 
}

In our experiments, we took real-world efficient and secure
low-level implementations of arithmetic operations over
$\bbfGF(\varrho)$ in~\todo{reference?}, manually translated source
codes to our domain specific language, specified their algebraic
properties, and performed certified verification with our technique. 
Table~\ref{table:arithmetic-operations} summarizes the results.


\begin{table}[ht]
  \caption{Certified Verification of Arithmetic Operations over
    $\bbfGF(\varrho)$}
  \centering
  \begin{tabular}{|c|c|c|c|}
    \hline
             & number of lines & time (seconds) & remark\\
    \hline
    addition & 10 &                       & $a \Fplus b$ \\
    \hline
    subtraction & 15 &                    & $a \Fminus b$ \\
    \hline
    multiplication & 169 &                & $a \Ftimes b$\\
    \hline
    multiplication by 121666 & 31 &       & $121666 \Ftimes a$\\
    \hline
    square & 124 &                        & $a \Ftimes a$\\
    \hline
  \end{tabular}
  \label{table:arithmetic-operations}
\end{table}

\todo{range check}

%%% Local Variables: 
%%% mode: latex
%%% eval: (TeX-PDF-mode 1)
%%% eval: (TeX-source-correlate-mode 1)
%%% TeX-master: "certified_vcg"
%%% End: 


\vspace{-.5em}
\subsection{The Montgomery Ladderstep}
\vspace{-.5em}
\label{subsection:evaluation:ladder-step}

\hide{
\todo{Bo-Yin, please describe how the ladder step is used and why it is 
important for about .5 page. Maybe another .5 page for its mathematical 
properties? }
}

Recall that X25519 is based on the elliptic curve Curve25519. 
Cryptographic primitives of the key exchange protocol perform
sequences of algebraic operations on the Abelian group $\bbfG$ induced
by Curve25519, not on the finite field $\bbfGF(\varrho)$. As
aforementioned, the binary operation $\Gplus$ requires another
sequence of arithmetic computation over $\bbfGF(\varrho)$. Errors
could still be introduced or even implanted in implementations of
$\Gplus$. Correctness of arithmetic constructs over $\bbfGF(\varrho)$
does not necessarily imply the correctness of the mathematical
construct in X25519. The Montgomery Ladderstep is the mathematical
construct widely used to implement $\Gplus$ on Curve25519. We 
verify a low-level implementation of the mathematical construct in
this experiment. 

\begin{algorithm}[h]
\label{evaluation:ladder-step:montgomery}
\begin{algorithmic}[1]
\Function{ladderstep}{$x_1, x_m, z_m, x_n, z_n$}
\begin{multicols}{3}
\State $t_1 \leftarrow x_m \Fplus z_m$
\State $t_2 \leftarrow x_m \Fminus z_m$
\State $t_7 \leftarrow t_2 \Ftimes t_2$
\State $t_6 \leftarrow t_1 \Ftimes t_1$
\State $t_5 \leftarrow t_6 \Fminus t_7$
\State $t_3 \leftarrow x_n \Fplus z_n$
\State $t_4 \leftarrow x_n \Fminus z_n$\rule{0ex}{0ex}
\State $t_9 \leftarrow t_3 \Ftimes t_2$
\State $t_8 \leftarrow t_4 \Ftimes t_1$
\State $x_n \leftarrow t_8 \Fplus t_9$
\State $z_n \leftarrow t_8 \Fminus t_9$
\State $x_n \leftarrow x_{n} \Ftimes x_{n}$
\State $z_n \leftarrow z_{n} \Ftimes z_{n}$
\State $z_n \leftarrow z_{n} \Ftimes x_1$\rule{0ex}{0ex} 
\State $x_m \leftarrow t_6 \Ftimes t_7$
\State $z_m \leftarrow 121666 \Ftimes t_5$
\State $z_m \leftarrow z_m \Fplus t_7$
\State $z_m \leftarrow z_m \Ftimes t_5$
\State \Return $(x_m, z_m, x_n, z_n)$
\EndFunction
\end{multicols}
\end{algorithmic}
\caption{Montgomery Ladderstep}
\end{algorithm}

Peter Montgomery derived a sequence of computations, now usually known
as "the Montgomery Ladderstep", to compute both $P+Q$ and $[2]Q$
simultaneously and efficiently knowing $\pm(P-Q)$.  The ladderstep can
be achieved only the x coordinates on a set of curves today known as
Montgomery curves. This is known as a differential addition chain and
is one of the fundamental ways to avoid timing attacks.  That is,
starting from points $[n]P$ and $[n+1]P$, one can get either $[2n]P$
and $[2n+1]P$, or $[2n+1]P$ and $[2n+2]P$ in a time-constant manner,
with the difference between the two points always remaining $\pm P$
using the Ladderstep, provided that there is a time-constant way to
swap the two (which there usually is).   This is equivalent to
computing either $2n$ or $2n+1$ from $n$ and is sufficient to compute
any $[n]P$ where n has a constant bitlength.   For curves not
equivalent to a Montgomery curve, a variant known as the "Co-Z ladder"
is available to effect the same differential addition chain. 

Montgomery's formulas and his steps are well known since his original
paper.  (include Montgomery ladderstep diagram and Montgomery's
original equations).  We show that the program carrys out a
computation equivalent to his formulas.  In fact, we show that the
doubling part of the formula has a coefficient of 4 in both the
numerator and denominator that is not in Peter Montgomery's formula
(insert cite here). 

Let $p \in G$ be an element of the group $\bbfG$ and hence a point on
the elliptic curve Curve25519. We write $n \cdot p$ for
$\overbrace{p \Gplus \cdots \Gplus p}^n \in G$ when $n \in \bbfN$. A
crucial mathematical construct in X25519 is to compute the
$x$-coordinate of $n \cdot p$ for $n \in \bbfN$. The na\"ive
implementation would follow the equation (\ref{eq:curve25519-group})
and compute both $x$- and $y$-coordinates with, in particular, a
division operation $\Fdiv$ in $\bbfGF(\varrho)$. In~\cite{M:87:SPEC},
a more efficient implementation is proposed. 

$np = x_n \Fdiv z_n$.
$\textsc{ladderstep} (x_1, x_m, z_m, x_n, z_n) = (x_{2m}, z_{2m},
x_{m+n}, z_{m+n})$
 
\todo{mention the coefficient $4$ in the ratio}

\vspace{-.5em}
\section{Conclusion}
\vspace{-.5em}
\label{section:conclusion}

We have developed techniques to verify algebraic specifications of
low-level mathematical constructs in cryptographic programs. Our case
studies on real low-level implementations of X25519 suggest the
applicability and scalability of our techniques. Currently, we are
working on certified techniques for range checks. We also plan to
apply our techniques to more low-level mathematical constructs in real
cryptographic programs.

\newpage

\bibliographystyle{splncs-sorted}
\bibliography{refs}

\end{document}

%%% Local Variables: 
%%% mode: latex
%%% eval: (TeX-PDF-mode 1)
%%% eval: (TeX-source-correlate-mode 1)
%%% TeX-master: "certified_vcg"
%%% End: 
