
\hide{
\todo{Bo-Yin, please describe how the ladder step is used and why it is 
important for about .5 page. Maybe another .5 page for its mathematical 
properties? }
}

Recall that X25519 is based on the elliptic curve Curve25519.
Cryptographic primitives of the key exchange protocol perform
sequences of algebraic operations on the Abelian group $\bbfG$ induced
by Curve25519, not on the finite field $\bbfGF(\varrho)$. As
aforementioned, the binary operation $\Gplus$ requires another
sequence of arithmetic computation over $\bbfGF(\varrho)$.
%
Correctness of arithmetic constructs over $\bbfGF(\varrho)$ does not
necessarily imply the correctness of the mathematical construct in
X25519. Errors could still be introduced or even implanted in
implementations of $\Gplus$.

%The Montgomery Ladderstep is the mathematical
%construct widely used to implement $\Gplus$ on Curve25519. We 
%verify a low-level implementation of the mathematical construct in
%this experiment. 
Let $P \in G$ be an element of the group $\bbfG$ and hence a point on
the elliptic curve Curve25519. We write $[n]P$ for
$\overbrace{P \Gplus \cdots \Gplus P}^n \in G$ when $n \in \bbfN$.
The operation we want is a ``point multiplication'' which is given $n$
and $P$ compute $[n]P$.  The na\"ive way to do this is ``double
and add''.  

\begin{algorithm}[h]
  \begin{algorithmic}[1]  
    \Function{point.multiple}{$m,P$}
    \begin{multicols}{2}
      \State $t\leftarrow P$
      \For{$i=\lfloor \log_2 m \rfloor-1\ \mathbf{downto}\ 1$}
      \State $t\leftarrow [2]P$
      \If{$($ bit $i$ of $n$ in binary $)=1$}
      \State $t\leftarrow t\Gplus P$
      \EndIf
      \EndFor
      \State \Return{$t$}
      \EndFunction
    \end{multicols}
  \end{algorithmic}
\end{algorithm}
Observe that given $[n]P$ before step 4, we first compute $[2n]P$
or $[2n+1]P$ by the end of step 7 by first doubling to
$[2n]P$ (step 4), then decide to add $P$ or not depending on a
whether a corresponding bit is 1.  But this is obviously not time
constant.  The obvious next step that we can double
(i.e. compute $[2n]P$), \emph{always add} $P$, then choose one
of the two using conditional moves or equivalents.  This however is
inefficient.  We need something equivalent to steps 4--7 but better.

\begin{algorithm}[h]
\label{evaluation:ladder-step:montgomery}
\begin{algorithmic}[1]
\Function{ladderstep}{$x_1, x_m, z_m, x_n, z_n$}
\begin{multicols}{3}
\State $t_1 \leftarrow x_m \Fplus z_m$
\State $t_2 \leftarrow x_m \Fminus z_m$
\State $t_7 \leftarrow t_2 \Ftimes t_2$
\State $t_6 \leftarrow t_1 \Ftimes t_1$
\State $t_5 \leftarrow t_6 \Fminus t_7$
\State $t_3 \leftarrow x_n \Fplus z_n$
\State $t_4 \leftarrow x_n \Fminus z_n$\rule{0ex}{0ex}
\State $t_9 \leftarrow t_3 \Ftimes t_2$
\State $t_8 \leftarrow t_4 \Ftimes t_1$
\State $x_n \leftarrow t_8 \Fplus t_9$
\State $z_n \leftarrow t_8 \Fminus t_9$
\State $x_n \leftarrow x_{n} \Ftimes x_{n}$
\State $z_n \leftarrow z_{n} \Ftimes z_{n}$
\State $z_n \leftarrow z_{n} \Ftimes x_1$\rule{0ex}{0ex} 
\State $x_m \leftarrow t_6 \Ftimes t_7$
\State $z_m \leftarrow 121666 \Ftimes t_5$
\State $z_m \leftarrow z_m \Fplus t_7$
\State $z_m \leftarrow z_m \Ftimes t_5$
\State \Return $(x_m, z_m, x_n, z_n)$
\EndFunction
\end{multicols}
\end{algorithmic}
\caption{Montgomery Ladderstep}
\end{algorithm}

Peter Montgomery derived a sequence of computations known
as "the Montgomery Ladderstep", to compute both $P_0+P_1$ and $[2]P_0$
simultaneously and efficiently knowing $P_1-P_0$.  The ladderstep can
be achieved only using $x$ coordinates on a set of curves today known as
Montgomery curves, which include Curve25519.  Furthermore, we can
avoid expensive divisions by using projective representations (i.e,
instead of $x$ write $x/z$ and work  with fractions).  

The Ladderstep starts from points $[n]P $ and
$[n+1]P $, to get either $ [2n]P $ and
$[2n+1]P$, or $[2n+1]P$ and $[2n+2]P$, the difference between the two points remaining $\pm P$.  Provided that there is a time-constant way to swap the two input
points, this
is equivalent to computing either $[2n]P$ or $[2n+1]P$
from $[n]P$ in constant
time and is sufficient to compute any $[n]P$ where
$n$ has a known bitlength.  This is known as a differential
addition chain and is one of the fundamental ways to avoid timing
attacks.  For curves not equivalent to a Montgomery curve, a variant
known as the "Co-$Z$ ladder" is available to effect the same
differential addition chain.

Montgomery's formulas and his steps are well known
since~\cite{M:87:SPEC}.  What we showed is that the program carrys out
a computation equivalent to his formulas\footnote{An amusing fact is
  that when we carry out the ladderstep, we find for ourselves the factor of $4$
  in both the numerator and denominator of the addition formulas, 
  noted on \cite[p.~261]{M:87:SPEC}.}.

% A
%crucial mathematical construct in X25519 is to compute the
%$x$-coordinate of $n \cdot p$ for $n \in \bbfN$. The na\"ive
%implementation would follow the equation (\ref{eq:curve25519-group})
%and compute both $x$- and $y$-coordinates with, in particular, a
%division operation $\Fdiv$ in $\bbfGF(\varrho)$. ,
%a more efficient implementation is proposed. 
%
%$np = x_n \Fdiv z_n$.
%$\textsc{ladderstep} (x_1, x_m, z_m, x_n, z_n) = (x_{2m}, z_{2m},
%x_{m+n}, z_{m+n})$
% 
%\todo{mention the coefficient $4$ in the ratio}
%

%%% Local Variables: 
%%% mode: latex
%%% eval: (TeX-PDF-mode 1)
%%% eval: (TeX-source-correlate-mode 1)
%%% TeX-master: "certified_vcg"
%%% End: 
