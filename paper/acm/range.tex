
First define the syntax of a fragment of QF\_BV with function names taken from the standard format SMTLIB2.
In this fragment, a variable always represents a bit-vector of width $\wordsize$ (the assumed wordsize).
Let $\qExp$ and $\qPred$ respectively denote the expressions and the predicates in QF\_BV.
An expression $e \in \qExp$ can be a constant $\bvconst(n, b)$, a variiable $v \in \Var$, an addition $\bvadd(e_1, e_2)$, a subtraction $\bvsub(e_1, e_2)$, a multiplication $\bvmul(e_1, e_2)$, a concatenation $\bvconcat(e_1, e_2)$, a zero extension $\bvext(e', i)$, a left-shifting $\bvshl(e_1, e_2)$, a logical right-shifting $\bvlshr(e_1, e_2)$, or an extraction $\bvextract(e', i, j)$ where $n, i, j \in \bbfN$, $b \in \BV^n$, and $e_1, e_2, e' \in \qExp$.
A predicate $q \in \qPred$ can be an equality $e_1 = e_2$, a less-than $\bvult(e_1, e_2)$, a less-then-or-equal $\bvule(e_1, e_2)$, a negation $\neg q'$, a conjunction $q_1 \wedge q_2$, or a disjunction $q_1 \vee q_2$ where $e_1, e_2 \in \qExp$ and $q', q_1, q_2 \in \qPred$.
An implication $q_1 \Rightarrow q_2$ is defined as $\neg q_1 \vee q_2$.

Let $\max(n, m)$ return the maximal number in $n$ and $m$.
Given an expression $e \in \qExp$, $\width(e)$ denotes the maximal bit-width of $e$.
\[
\begin{array}{rcl}
\width(\bvconst(n, b)) & = & n \\
\width(v) & = & w \\
\width(\bvadd(e_1, e_2)) & = & max(\width(e_1), \width(e_2))\\
\width(\bvsub(e_1, e_2)) & = & max(\width(e_1), \width(e_2))\\
\width(\bvmul(e_1, e_2)) & = & max(\width(e_1), \width(e_2))\\
\width(\bvconcat(e_1, e_2)) & = & \width(e_1) +_\bbfN \width(e_2)\\
\width(\bvext(e, i)) & = & \width(e) +_\bbfN i \\
\width(\bvshl(e_1, e_2)) & = & \width(e_1) \\
\width(\bvlshr(e_1, e_2)) & = & \width(e_1) \\
\width(\bvextract(e, i, j)) & = & i -_\bbfN j +_\bbfN 1 \\
\end{array}
\]
The expression $e$ is called \emph{well-formed} if $e$ is (1) a constant, a variable, a concatenation, a zero extension, a left-shifting, or a logical right-shifting, (2) an addition $\bvadd(e_1, e_2)$, a subtraction $\bvsub(e_1, e_2)$, or a multiplication $\bvmul(e_1, e_2)$ with $\width(e_1) = \width(e_2)$ and both $e_1$ and $e_2$ well-formed, or (3) an extraction $\bvextract(e', i, j)$ with $0 \leq j \leq i < \width(e')$ and $e'$ well-formed.
A predicate $q \in \qPred$ is well-formed if all sub-expressions are well-formed.

Let $\bvst \in \bvSt$ be a state.
Define the semantic function $\bvvalueof{e}(\bvst)$ for well-formed expressions $e$.
For a predicate $q \in \qPred$, we write $\BV^{\wordsize} \models q[\bvst]$ if $q$ evaluates to $\btt$ using the evaluation function $\bvvalueof{e}(\bvst)$ for every subexpression $e$ in $q$, using $<_\BV$ for $\bvult$, and using $\leq_\BV$ for $\bvule$.
\[
\begin{array}{rcl}
\bvvalueof{\bvconst(n, b)}(\bvst) & \defn & b \\
\bvvalueof{v}(\bvst) & \defn & \zvalueof{v}(\bvst) \\
\bvvalueof{\bvadd(e_1, e_2)}(\bvst) & \defn & \bvvalueof{e_1}(\bvst) +_\BV \bvvalueof{e_2}(\bvst) \\
\bvvalueof{\bvsub(e_1, e_2)}(\bvst) & \defn & \bvvalueof{e_1}(\bvst) -_\BV \bvvalueof{e_2}(\bvst) \\
\bvvalueof{\bvmul(e_1, e_2)}(\bvst) & \defn & \bvvalueof{e_1}(\bvst) \times_\BV \bvvalueof{e_2}(\bvst) \\
\bvvalueof{\bvconcat(e_1, e_2)}(\bvst) & \defn & \bvvalueof{e_1}(\bvst) ._\BV \bvvalueof{e_2}(\bvst) \\
\bvvalueof{\bvext(e, i)}(\bvst) & \defn & \bvvalueof{e}(\bvst) \#_\BV i \\
\bvvalueof{\bvshl(e_1, e_2)}(\bvst) & \defn & \bvvalueof{e_1}(\bvst) \ll_\BV |\bvvalueof{e_2}(\bvst)| \\
\bvvalueof{\bvlshr(e_1, e_2)}(\bvst) & \defn & \bvvalueof{e_1}(\bvst) \gg_\BV |\bvvalueof{e_2}(\bvst)| \\
\bvvalueof{\bvextract(e, i, j)}(\bvst) & \defn & \bvvalueof{e}(\bvst)[i, j] \\
\end{array}
\]

%Now we are ready to convert overflow/underflow checks and range problems to QF\_BV.
Let $q_r, q_r' \in \bvPredr$ be two range predicates and $p \in \bvProg$ a well-formed program in static single assignment form.
Both an overflow/underflow check ($\BV^\wordsize \models q_r[\bvst]$ implies $\textsc{ProgSafe}(p, \bvst) = \btt$ for all $\bvst \in \bvSt$) and a range problem ($\models \hoaretriple{q_r}{p}{q_r'}$) involve only bit-vector operations and can be easily modeled by QF\_BV expressions according to the semantics of $\textsc{ProgSafe}$, programs, range predicates, and range specifications.
To save space, we only show some examples.
Define $\textsc{AtomToQFBV}(a)$ as $v$ when the atom $a$ is a variable $v$ and otherwise $\bvconst(w, b)$ when $a$ is a constant $b$.
Let $\textsc{ProgToQFBV}(p)$ be the corresponding formula of $p$ in QF\_BV such that $\bvst \goesto{p} \bvst'$ implies $\BV^\wordsize \models \textsc{ProgToQFBV}(p)[\bvst']$.
For example, $\textsc{ProgToQFBV}($$v$ $\leftarrow$ $a_1$ $+$ $a_2)$ is $v$ $=$ $\textsc{AtomToQFBV}(a_1)$ $+_\BV$ $\textsc{AtomToQFBV}(a_2)$.
%$\textsc{ProgToQFBV}($$v_h\ v_l \leftarrow a @ n)$ is $v_h$ $=$ $\textsc{AtomToQFBV}(a)$ $\gg_\BV$ $\bv^w(n)$ $\wedge$ $v_l$ $=$ $(\textsc{AtomToQFBV}(a)$ $\ll_\BV$ $\bv^w(w - n))$ $\gg_\BV$ $\bv^w(w - n)$
Let $\textsc{SafeToQFBV}(p)$ be the corresponding formula for $\textsc{ProgSafe}$ such that $\textsc{ProgSafe}(p, \bvst) = \btt$ if and only if $\BV^\wordsize \models \textsc{SafeToQFBV}(p)$.
Let $\textsc{PredrToQFBV}(q_r)$ be the corresponding formula for a $q_r \in \bvPredr$ such that $\BV^\wordsize \models q_r$ if and only if $\BV^\wordsize \models \textsc{PredrToQFBV}(q_r)$.
The following theorem states the soundness of our translation from range problems and overflow/underflow checks to QF\_BV.

\begin{theorem}
Given two range predicates $q_r, q_r' \in \bvPredr$ and a well-formed program $p \in \bvProg$ in static single assignment form,
\begin{itemize}
\item $\BV^\wordsize \models q_r[\bvst]$ implies $\textsc{ProgSafe}(p, \bvst) = \btt$ for all $\bvst \in \bvSt$ if, $(\textsc{PredrToQFBV}(q_r) \wedge \textsc{ProgToQFBV}(p)) \Rightarrow \textsc{SafeToQFBV}(p)$ is valid, and
\item $\models \hoaretriple{q_r}{p}{q_r'}$ if $\textsc{PredrToQFBV}(q_r) \wedge \textsc{ProgToQFBV}(p) \Rightarrow \textsc{PredrToQFBV}(q_r')$ is valid.
\end{itemize}
\label{theorem:to-qfbv}
\end{theorem}





