 
The operation $\Gplus$ is defined by arithmetic computation over
$\bbfF$. %In order to compute $P_0 \Gplus P_1$ for $P_0, P_1 \in G$,
Mathematical constructs for
arithmetic over $\bbfF$ are hence necessary. Recall that
an element in $\bbfF$ is represented by a 256-bit number. 
% The na\"ive implementation for arithmetic operations over $\bbfF$
% would require rithmetic computation over $\bbfZ$. 
Arithmetic computation for 255-bit integers however is not yet
available in commodity computing devices as of the year
2017; it has to be carried out by limbs where a
\emph{limb} is a 32- or 64-bit number depending on the underlying
computer architectures. Figure~\ref{figure:dsl:subtraction}
(Section~\ref{section:domain-specific-language-bvdsl}) is such an
implementation of subtraction for the AMD64 architecture. 


\hide{
In practice, efficient long-integer arithmetic however is more
complicated. Consider subtracting a long integer from another. The
na\"ive implementation would simply subtract by limbs, propagate carry
flags, and add the prime number $\varrho$ if necessary. 
Execution time of the na\"ive subtraction however varies when minuend
is less than subtrahend. It thus allows timing attacks and is 
insecure. The na\"ive implementation of 255-bit subtraction should
never be used in cryptographic programs.
}

Multiplication is another interesting but much more
complicated computation. The na\"ive implementation for 255-bit
multiplication would compute a 510-bit product and then find the
corresponding 255-bit representation by division. 
An efficient implementation for 255-bit multiplication avoids
division by performing modulo operations aggressively. 
For instance, an intermediate
result of the form $c \Ftimes 2^{255}$ is immediately replaced by
$c \Ftimes 19$ since $2^{255} \Feq 19$ in $\bbfGF(\varrho)$. 
This is indeed how the most efficient multiplication for the AMD64
architecture is implemented (Appendix~\ref{appendix:multiplication})~\cite{BDL+:11:HSHSS,BDL+:12:HSHSS}.

\hide{
Recall that 
elements in $\bbfF$ are represented by five limbs of 51-bit unsigned
integers. Consider multiplying two 255-bit values
\begin{equation*}
  \begin{array}{lcccccccccccl}
    x & = & \mathit{radix51} (x_4, x_3, x_2, x_1, x_0) & = &
            2^{51 \times 4} x_4 & + & 2^{51 \times 3} x_3 & + &
            2^{51 \times 2} x_2 & + & 2^{51 \times 1} x_1 & + &
            2^{51 \times 0} x_0 \textmd{ and}\\
    y & = & \mathit{radix51} (y_4, y_3, y_2, y_1, y_0) & = &
            2^{51 \times 4} y_4 & + & 2^{51 \times 3} y_3 & + &
            2^{51 \times 2} y_2 & + & 2^{51 \times 1} y_1 & + &
            2^{51 \times 0} y_0.
  \end{array}
\end{equation*}
The intermediate results $2^{255} x_4 y_1$, $2^{255} x_3 y_2$,
$2^{255} x_2 y_3$, and $2^{255} x_1 y_4$ can all be replaced by 
$19 x_4 y_1$, $19 x_3 y_2$, $19 x_2 y_3$, and $19 x_1 y_4$,
respectively. Division is not needed in such implementations.
}

\hide{
We describe but a couple of optimizations in efficient implementations
of arithmetic operations over $\bbfF$. Real-world implementations
are necessarily optimized with various algebraic properties in
$\bbfF$. 
}

In our experiment, we took real-world efficient and secure
low-level implementations of arithmetic computation over
$\bbfGF(\varrho)$ from~\cite{BDL+:11:HSHSS,BDL+:12:HSHSS},
manually translated source
codes to our domain specific language, specified their algebraic and range
properties, and performed certified verification with our technique.
Table~\ref{table:arithmetic-operations} summarizes the results
without and with applying the two heuristics in Section~\ref{subsection:solving-algebraic-equations}.
The column ``overflow'' shows the time used by \boolector to verify if the input program overflows.
The column ``range'' shows the time used by \boolector to verify the range specification of the input program.
The columns ``algebraic'' show the time used by \singular to verify the algebraic specification of the input program.
The columns ``total'' show the total verification time including overflow check, verification of range and algebraic specifications, rewriting, proof certification, etc.
The columns ``without heuristics'' and ``with heuristics'' respectively show the time information without and with the two heuristics.
The results show that without the two heuristics, multiplication and square
cannot be verified because the computation of Gr\"obner bases was killed by the OS
after running for days.
With the heuristics, all the implementations can be verified in seconds.



\begin{table*}[ht]
  \caption{Certified Verification of Arithmetic Operations over
    $\bbfGF(\varrho)$}
  \centering
  \begin{tabular}{|c|c|c|c|c|c|c|c|c|}
    \hline
    \multirow{3}{*}{} & \multirow{3}{*}{number of lines} & \multicolumn{6}{|c|}{time (seconds)} & \multirow{3}{*}{remark} \\ \cline{3-8}
    & & \multirow{2}{*}{overflow} & \multirow{2}{*}{range} & \multicolumn{2}{|c|}{without heuristics} & \multicolumn{2}{|c|}{with heuristics} & \\ \cline{5-8}
    & & & & algebraic & total & algebraic & total & \\
    \hline
    addition                 & 10  & 0.162 & 0.249 & 30.545 & 41.55 & 0.401 & 4.14 & $a \Fplus b$ \\
    \hline
    subtraction              & 15  & 0.140 & 0.389 & 35.646 & 48.47 & 0.208 & 4.93 & $a \Fminus b$ \\
    \hline
    multiplication           & 144 & 3.904 & 41.070 & - & - & 2.312 & 81.93 & $a \Ftimes b$\\
    \hline
    multiplication by 121666 & 26  & 0.266 & 0.852 & 1112.311 & 1125.41 & 0.315 & 7.70 & $121666 \Ftimes a$\\
    \hline
    square                   & 109 & 3.722 & 19.905  & - & - & 1.087 & 47.44 & $a \Ftimes a$\\
    \hline
  \end{tabular}
  \label{table:arithmetic-operations}
\end{table*}

% {range check}

We also perform range checks and program overflow checks with the SMT solver \boolector on the
five mathematical constructs with results shown in Table~\ref{table:arithmetic-operations}~\cite{boolector}. The most complicated case (multiplication) takes 19.82 seconds.
% Compared to algebraic specifications, verification time for range
% check is insignificant.
Although our range check and program overflow check are not yet certified, low-level
mathematical constructs in X25519 are formally verified automatically.

%%% Local Variables:
%%% mode: latex
%%% eval: (TeX-PDF-mode 1)
%%% eval: (TeX-source-correlate-mode 1)
%%% TeX-master: "certified_vcg"
%%% End:
