
Consider any predicates $q_a, q_a' \in \bvPreda$, $q_r, q_r' \in \bvPredr$, and well-formed program $p \in
\bvProg$. Let $\theta_0 (v) = 0$ for every $v \in \Var$. By
Theorem\ref{theorem:ssa}, \ref{theorem:bv2z}, and \ref{theorem:program-to-q-soundness}, we have
\begin{equation*}
  \begin{array}{cll}
    & \models \hoaretriple{q_a \wedge q_r}{p}{q_a' \wedge q_r'}\\
%    \Leftrightarrow
%    & \models
%      \hoaretriple{q}{\textsc{SliceProg}(\textsc{VarsInPred}(q'), p)}{q'}
%    & \textmd{ (Theorem~\ref{theorem:program-slicing})}\\
    \Leftrightarrow
    & \models
      \hoaretriple{\hat{q_a} \wedge \hat{q_r}}
      {\hat{p}}
      {\hat{q_a'} \wedge \hat{q_r'}}
    & \textmd{ (Theorem~\ref{theorem:ssa})}\\
    &
      \textmd{where } \langle \hat{\theta}, \hat{p} \rangle =
      \textsc{SSAProg} (\theta_0, p), \\
    & \hat{q_a} = \textsc{SSAPred}(\theta_0, q_a), \\
    & \hat{q_r} = \textsc{SSAPred}(\theta_0, a_r), \\
    & \hat{q_a'} = \textsc{SSAPred}(\hat{\theta}, q_a'), \textmd{ and} \\
    & \hat{q_r'} = \textsc{SSAPred}(\hat{\theta}, q_r') \\
    \Leftarrow
    & \textmd{program $\hat{p}$ does not overflow/underflow}, \\
    & \models \hoaretriple{\hat{q_r}}{\hat{p}}{\hat{q_r'}}, \textmd{ and } \\
    & \models \hoaretriple{\tilde{q_a}}{\tilde{p}}{\tilde{q_a'}}
    & \textmd{ (Theorem~\ref{theorem:bv2z})} \\
    & \textmd{where } \tilde{p} = \textsc{BV2ZProg}(\hat{p}), \\
    & \tilde{q_a} = \textsc{BV2ZExpa}(\hat{q_a}), \textmd{ and} \\
    & \tilde{q_a'} = \textsc{BV2ZExpa}(\hat{q_a'}) \\
    \Leftarrow
    & \textmd{program $\hat{p}$ does not overflow/underflow}, \\
    & \models \hoaretriple{\hat{q_r}}{\hat{p}}{\hat{q_r'}}, \textmd{ and } \\
    & \bbfZ \models \forall \vx.
      \Pi (\hoaretriple{\tilde{q_a}}{\tilde{p}}{\tilde{q_a'}})
    & \textmd{ (Theorem~\ref{theorem:program-to-q-soundness})}
  \end{array}
\end{equation*}
Observe that $\tilde{p}$ is well-formed and in static single assignment
form (Lemma~\ref{lemma:ssa-programs} and \ref{lemma:bv2z-prog}).
Theorem~\ref{theorem:program-to-q-soundness} is applicable in the last
deduction. After the translations, an instance of the modular
polynomial equation entailment problem is obtained from the given
algebraic specification of a well-formed program in \bvdsl.
To verify mathematical constructs against their algebraic
specifications, we will solve the entailment problem.