
A program is in \emph{static single assignment} form if every non-input variable
is assigned at most once and no input variable is assigned~\cite{AWZ:88:DQVP}.
Our next task is to transform any specification
$\hoaretriple{q}{p}{q'}$ to a specification of $p$ in
static single  assignment form for any $q, q' \in \bvPred$ and $p \in
\bvProg$.
To avoid ambiguity, we consider \emph{well-formed} programs where
\begin{itemize}
\item for every statement in the program with two lvalues such as $c\ v \leftarrow a_1 + a_2 + y$ with lvalues $c$ and $v$, the two lvalues are not equal; and
\item every non-input program variable must be assigned to a value
  before being used.
\end{itemize}

Our transformation maintains a finite mapping $\theta$ from variables to
non-negative integers. For any variable $v$, $v^{\theta(v)}$ is
the most recently assigned copy of $v$.
For any atom $a$, $a^{\theta}$ is $v^{\theta(v)}$ when $a$ is a variable $v$, and otherwise is $b$ when $a$ is a constant bit-vector $b$.
Only the most recent copies of variables are referred in
expressions. Algorithm~\ref{algorithm:ssa-expressions}
transforms algebraic expressions with the finite mapping $\theta$ by
structural induction. Integers are unchanged. For each variable, its
most recent copy is returned by looking up the mapping $\theta$. Other
algebraic expressions are transformed recursively.

\begin{algorithm}
  \begin{algorithmic}[1]
    \Function{SSAExpra}{$\theta$, $e$}
    \Match{$e$}
      \Case{$i$} \Return $i$ \EndCase
      \Case{$v$} \Return $v^{\theta(v)}$ \EndCase
      \Case{$-e'$} \Return $-$\Call{SSAExpra}{$\theta$, $e'$} \EndCase
      \Case{$e_1 + e_2$}
        \State{\Return \Call{SSAExpra}{$\theta$, $e_1$} $+$ \Call{SSAExpra}{$\theta$, $e_2$}}
      \EndCase
      \Case{$e_1 - e_2$}
        \State{\Return \Call{SSAExpra}{$\theta$, $e_1$} $-$ \Call{SSAExpra}{$\theta$, $e_2$}}
      \EndCase
      \Case{$e_1 \times e_2$}
        \State{\Return \Call{SSAExpra}{$\theta$, $e_1$} $\times$ \Call{SSAExpra}{$\theta$, $e_2$}}
      \EndCase
    \EndMatch
    \EndFunction
  \end{algorithmic}
  \caption{Static Single Assignment Transformation for Algebraic Expressions}
  \label{algorithm:ssa-expressions}
\end{algorithm}

Similarly, algebraic and range predicates must refer to most recent copies of
variables. They are transformed according to the finite mapping
$\theta$. Thanks to the formalization of finite mappings in \coq. Both
algorithms are easily specified in \gallina.
Let $\textsc{SSAPreda}$ and $\textsc{SSAPredr}$ denote the transformations for $\bvPreda$ and $\bvPredr$ respectively.
The function $\textsc{SSAPred}$ then transforms the algebraic part and the range part of a predicate separatedly with $\textsc{SSAPreda}$ and $\textsc{SSAPredr}$, that is, given $q_a \in \bvPreda$, $q_r \in \bvPredr$, and a mapping $\theta$, $\textsc{SSAPred}(\theta, q_a \andar q_r)$ $\defn$ $\textsc{SSAPreda}(\theta, q_a) \andar \textsc{SSAPredr}(\theta, q_r)$.

\hide{
\begin{algorithm}
  \begin{algorithmic}[1]
    \Function{SSAPreda}{$\theta$, $q$}
    \Match{$q$}
      \Case{$\top$} \Return $\top$ \EndCase
      \Case{$e' = e''$}
        \Return \Call{SSAExpra}{$\theta$, $e$} = \Call{SSAExpra}{$\theta$, $e'$}
      \EndCase
      \Case{$e' \equiv e'' \mod n$}
        \Return \Call{SSAExpra}{$\theta$, $e$} $\equiv$
                \Call{SSAExpra}{$\theta$, $e'$} $\mod n$
      \EndCase
      \Case{$q' \wedge q''$}
        \Return \Call{SSAPreda}{$\theta$, $q'$} $\wedge$
                \Call{SSAPreda}{$\theta$, $q''$}
      \EndCase
    \EndMatch
    \EndFunction
  \end{algorithmic}
  \caption{Static Single Assignment Transformation for Algebraic Predicates}
  \label{algorithm:ssa-predicates}
\end{algorithm}
}

\hide{
\begin{algorithm}
  \begin{algorithmic}[1]
    \Function{SSAAtom}{$\theta$, $a$}
    \Match{$a$}
      \Case{$b$}
        \Return $b$
      \EndCase
      \Case{$v$}
        \Return $v^{\theta(v)}$
      \EndCase
    \EndMatch
    \EndFunction
  \end{algorithmic}
  \caption{Static Single Assignment Transformation for Atoms}
  \label{algorithm:ssa-atoms}
\end{algorithm}
}

\begin{algorithm}
  \begin{algorithmic}[1]
    \Function{SSAStmt}{$\theta$, $s$}
    \Match{$s$}
      \Case{$v \leftarrow a$}
        \State{$\theta' \leftarrow \theta\mymapsto{v}{\theta(v) + 1}$}
        \State{\Return $\langle \theta'$,
                $v^{\theta'(v)} \leftarrow$ $a^\theta$$\rangle$}
      \EndCase
      \Case{$v \leftarrow a_1 + a_2$}
        \State{$\theta' \leftarrow \theta\mymapsto{v}{\theta(v) + 1}$}
        \State{\Return $\langle \theta'$,
                $v^{\theta'(v)} \leftarrow$ $a_1^\theta$ $+$ $a_2^\theta$$\rangle$}
      \EndCase
      \Case{$c\ v \leftarrow a_1 + a_2$}
        \State{$\theta' \leftarrow \theta\mymapsto{c}{\theta(c) + 1}$}
        \State{$\theta'' \leftarrow \theta'\mymapsto{v}{\theta(v) + 1}$}
        \State{\Return $\langle \theta''$,
                $c^{\theta'(c)} \ v^{\theta''(v)} \leftarrow$ $a_1^\theta$ $+$ $a_2^\theta$$\rangle$}
      \EndCase
      \Case{$v \leftarrow a_1 + a_2 + y$}
        \State{$\theta' \leftarrow \theta\mymapsto{v}{\theta(v) + 1}$}
        \State{\Return $\langle \theta'$,
                $v^{\theta'(v)} \leftarrow$ $a_1^\theta$ $+$ $a_2^\theta$ $+$ $y^{\theta(y)}$$\rangle$}
      \EndCase
      \Case{$c\ v \leftarrow a_1 + a_2 + y$}
        \State{$\theta' \leftarrow \theta\mymapsto{c}{\theta(c) + 1}$}
        \State{$\theta'' \leftarrow \theta'\mymapsto{v}{\theta(v) + 1}$}
        \State{\Return $\langle \theta''$,
                $c^{\theta'(c)} \ v^{\theta''(v)} \leftarrow$ $a_1^\theta$ $+$ $a_2^\theta$ $+$ $y^{\theta(y)}$$\rangle$}
      \EndCase
      \Case{$v \leftarrow a_1 - a_2$}
        \State{$\theta' \leftarrow \theta\mymapsto{v}{\theta(v) + 1}$}
        \State{\Return $\langle \theta'$,
                $v^{\theta'(v)} \leftarrow$ $a_1^\theta$ $-$ $a_2^\theta$$\rangle$}
      \EndCase
      \Case{$v \leftarrow a_1 \times a_2$}
        \State{$\theta' \leftarrow \theta\mymapsto{v}{\theta(v) + 1}$}
        \State{\Return $\langle \theta'$,
                $v^{\theta'(v)} \leftarrow$ $a_1^\theta$ $\times$ $a_2^\theta$$\rangle$}
      \EndCase
      \Case{$v_h\ v_l \leftarrow a_1 \times a_2$}
        \State{$\theta' \leftarrow \theta\mymapsto{v_h}{\theta(v_h) + 1}$}
        \State{$\theta'' \leftarrow \theta'\mymapsto{v_l}{\theta(v_l) + 1}$}
        \State{\Return $\langle \theta''$,
                $v_h^{\theta'(v_h)} \ v_l^{\theta''(v_l)} \leftarrow$ $a_1^\theta$ $\times$ $a_2^\theta$$\rangle$}
      \EndCase
      \Case{$v \leftarrow a \ll n$}
        \State{$\theta' \leftarrow \theta\mymapsto{v}{\theta(v) + 1}$}
        \State{\Return $\langle \theta'$,
                $v^{\theta'(v)} \leftarrow$ $a^\theta$ $\ll$ $n$$\rangle$}
      \EndCase
      \Case{$v_h\ v_l \leftarrow a@n$}
        \State{$\theta' \leftarrow \theta\mymapsto{v_h}{\theta(v_h) + 1}$}
        \State{$\theta'' \leftarrow \theta'\mymapsto{v_l}{\theta(v_l) + 1}$}
        \State{\Return $\langle \theta''$,
                $v_h^{\theta'(v_h)} \ v_l^{\theta''(v_l)} \leftarrow$ $a^\theta$ $@$ $n$$\rangle$}
      \EndCase
      \Case{$v_h\ v_l \leftarrow (a_1.a_2) \ll n$}
        \State{$\theta' \leftarrow \theta\mymapsto{v_h}{\theta(v_h) + 1}$}
        \State{$\theta'' \leftarrow \theta'\mymapsto{v_l}{\theta(v_l) + 1}$}
        \State{\Return $\langle \theta''$,
                $v_h^{\theta'(v_h)} \ v_l^{\theta''(v_l)} \leftarrow$ ($a_1^\theta$ $.$ $a_2^\theta$) $@$ $n$$\rangle$}
      \EndCase
    \EndMatch
    \EndFunction
  \end{algorithmic}
  \caption{Static Single Assignment Transformation for Statements}
  \label{algorithm:ssa-statements}
\end{algorithm}

Statement transformation is slightly more complicated
(Algorithm~\ref{algorithm:ssa-statements}). For
atoms and variables on the right hand side, they are transformed by the given
finite mapping $\theta$.
The algorithm of statement transformation then updates $\theta$ and
replaces assigned variables with their latest copies.
%In the algorithm, $\theta\mymapsto{v}{n}$ denotes updating the finite
%mapping $\theta$ by abusing the notation.

It is straightforward to transform programs to static single
assignment form (Algorithm~\ref{algorithm:ssa-programs}). Using the
initial mapping $\theta_0$ from variables to $0$,
the algorithm starts from the first statement and obtains an
updated mapping with the statement in static single assignment form. It
continues to transform the next statement with the updated mapping.
Note that our algorithm works for any initial mapping but we choose a specific one to simplify our Coq proof.

\begin{figure*}
  \centering
  $
  \begin{array}{lclcl}
    \begin{array}{rcl}
    \textmd{1:} && {r}^1_0 \leftarrow {x}^0_0; \\
    \textmd{2:} && {r}^1_1 \leftarrow {x}^0_1; \\
    \textmd{3:} && {r}^1_2 \leftarrow {x}^0_2; \\
    \textmd{4:} && {r}^1_3 \leftarrow {x}^0_3; \\
    \textmd{5:} && {r}^1_4 \leftarrow {x}^0_4; \\
    \end{array}
    &\hspace{.05\textwidth}&
    \begin{array}{rcl}
    \textmd{6:} &&
      {r}^2_0 \leftarrow {r}^1_0 + {0xFFFFFFFFFFFDA}; \\
    \textmd{7:} &&
      {r}^2_1 \leftarrow {r}^1_1 + {0xFFFFFFFFFFFFE}; \\
    \textmd{8:} &&
      {r}^2_2 \leftarrow {r}^1_2 + {0xFFFFFFFFFFFFE}; \\
    \textmd{9:} &&
      {r}^2_3 \leftarrow {r}^1_3 + {0xFFFFFFFFFFFFE}; \\
    \textmd{10:} &&
      {r}^2_4 \leftarrow {r}^1_4 + {0xFFFFFFFFFFFFE};\\
    \end{array}
    &\hspace{.05\textwidth}&
    \begin{array}{rcl}
    \textmd{11:} && {r}^3_0 \leftarrow {r}^2_0 - {y}^0_0; \\
    \textmd{12:} && {r}^3_1 \leftarrow {r}^2_1 - {y}^0_1; \\
    \textmd{13:} && {r}^3_2 \leftarrow {r}^2_2 - {y}^0_2; \\
    \textmd{14:} && {r}^3_3 \leftarrow {r}^2_3 - {y}^0_3; \\
    \textmd{15:} && {r}^3_4 \leftarrow {r}^2_4 - {y}^0_4;
    \end{array}
  \end{array}
  $
  \caption{Subtraction \dslcode{bSubSSA} in Static Single Assignment Form}
% $(\dslcode{x}_0, \dslcode{x}_1, \dslcode{x}_2, \dslcode{x}_3,
% \dslcode{x}_4, \dslcode{y}_0, \dslcode{y}_1, \dslcode{y}_2,
% \dslcode{y}_3, \dslcode{y}_4)$
  \label{figure:translation:subtraction-ssa}
\end{figure*}

\begin{algorithm}
  \begin{algorithmic}[1]
    \Function{SSAProg}{$\theta$, $p$}
    \Match{$p$}
      \Case{$\epsilon$}
        \Return $\langle \theta,$ $\epsilon \rangle$
      \EndCase
      \Case{$s; pp$}
        \State{$\langle \theta'$, $s' \rangle$ $\leftarrow$
                 \Call{SSAStmt}{$\theta$, $s$}}
        \State{$\langle \theta''$, $pp'' \rangle$ $\leftarrow$
                 \Call{SSAProg}{$\theta'$, $pp$}}
        \State{\Return $\langle \theta''$, $s'; pp'' \rangle$}
      \EndCase
    \EndMatch
    \EndFunction
  \end{algorithmic}
  \caption{Static Single Assignment for Programs}
  \label{algorithm:ssa-programs}
\end{algorithm}

Using the specifications of Algorithm~\ref{algorithm:ssa-statements}
and~\ref{algorithm:ssa-programs} in \gallina, properties of these
algorithms are formally proven in \coq. We first show that
Algorithm~\ref{algorithm:ssa-programs} preserves well-formedness and
produces a program in static single assignment form.
\begin{lemma}
  Let $p \in \bvProg$ a well-formed program. If
  $\langle \hat{\theta}, \hat{p} \rangle$ $=
  \textsc{SSAProg}(\theta_0, p)$, then
  $\hat{p}$ is well-formed and in static single assignment form.
  \label{lemma:ssa-programs}
\end{lemma}

The next theorem shows that our transformation is both sound and
complete. That is, a specification is valid if and only if its
corresponding specification in static single assignment form is valid.
\begin{theorem}
  For every $q, q' \in \bvPred$ and $p \in \bvProg$,
  \begin{center}
    $\models \hoaretriple{q}{p}{q'}$ if and only if
    $\models \hoaretriple{\textsc{SSAPred}(\theta_0, q)}
    {\hat{p}}
    {\textsc{SSAPred}(\hat{\theta}, q')}$
  \end{center}
  where $\langle \hat{\theta}, \hat{p} \rangle =
  \textsc{SSAProg}(\theta_0, p)$.
  \label{theorem:ssa}
\end{theorem}

\noindent
\emph{Example.}
Figure~\ref{figure:translation:subtraction-ssa} gives the subtraction program
\dslcode{bSub} in static single assignment form. Starting from $0$, the
index of a variable is incremented when the variable is assigned to an
expression. After the static single assignment translation, the
variables ${x}_i$'s, ${y}_i$'s are indexed by $0$ and
${r}_i$'s are indexed by $3$ for $0 \leq i \leq 4$.
Subsequently, variables in pre- and post-conditions of the
specification for subtraction need to be indexed.
Let $\hat{q}_a$ $\defn$ $\top$, $\hat{q}_r$ $\defn$ $0$ $\leq$ ${x}^0_0,$ ${x}^0_1,$ ${x}^0_2,$ ${x}^0_3,$ ${x}^0_4,$ ${y}^0_0,$ ${y}^0_1,$ ${y}^0_2,$ ${y}^0_3,$ ${y}^0_4$ $\leq$ $\bv^{64}(2^{51} +_\bbfZ 2^{15})$, $\hat{q}_a'$ $\defn$ $\mathit{radix51}_\BV({x}^0_4, {x}^0_3, {x}^0_2, {x}^0_1, {x}^0_0)$ $-$ $\mathit{radix51}_\BV({y}^0_4, {y}^0_3, {y}^0_2, {y}^0_1, {y}^0_0)$ $\equiv$ $\mathit{radix51}_\BV({r}^3_4, {r}^3_3, {r}^3_2, {r}^3_1, {r}^3_0)$ $\mod$ $\myprime$, and $q_r'$ $\defn$ $0$ $\leq$ ${r}^3_0,$ ${r}^3_1,$ ${r}^3_2,$ ${r}^3_3,$ ${r}^3_4$ $<$ $\bv^{64}(2^{54})$.
The corresponding specification of in static single assignment form is then
\[
\cond{\hat{q}_a \wedge \hat{q}_r} \dslcode{bSubSSA} \cond{\hat{q}_a' \wedge \hat{q}_r'}.
\]

\hide{
\[
\begin{array}{rcl}
\cond{\hat{q}_a \wedge \hat{q}_r} &
\dslcode{bSubSSA} &
\cond{\hat{q}_a' \wedge \hat{q}_r'}.
\end{array}
\]
}
