
%By Theorem~\ref{theorem:program-to-q-soundness}, 
It remains to show 
\[
\begin{array}{l}
  \bbfZ \models \forall \vx. % \in \bbfZ^{|\vx|}.
  \bigwedge\limits_{i \in [I]} e_i (\vx) = e'_i (\vx) \wedge
  \bigwedge\limits_{j \in [J]} f_j (\vx) \equiv f'_j (\vx) \mod n_j
  \implies
  \\
  \hspace{.3\textwidth}
  \bigwedge\limits_{k \in [K]} g_k (\vx) = g'_k (\vx) \wedge
  \bigwedge\limits_{l \in [L]} h_l (\vx) \equiv h'_l (\vx) \mod m_l
\end{array}
\]
where
$e_i (\vx), e'_i (\vx), f_j (\vx), f'_j (\vx),
 g_k (\vx), g'_k (\vx), h_l (\vx), h'_l (\vx) \in
 \bbfZ[\vx]$, $n_j, m_l \in \bbfN$ for $i \in [I]$, $j \in [J]$, $k
 \in [K]$, and $l \in [L]$. Since the
consequence is a conjunction of (modular) equations, it suffices to 
prove one conjunct at a time. That is, we aim to show
\begin{equation*}
  \label{eq:polynomial-equation-implicant}
  \bbfZ \models \forall \vx. % \in \bbfZ^{|\vx|}.
  \bigwedge\limits_{i \in [I]} e_i (\vx) = e'_i (\vx) \wedge
  \bigwedge\limits_{j \in [J]} f_j (\vx) \equiv f'_j (\vx) \mod n_j
  \implies
  g (\vx) = g' (\vx); \textmd{ or}
\end{equation*}
 \begin{equation*}
   \label{eq:modular-polynomial-equation-implicant}
   \bbfZ \models \forall \vx. % \in \bbfZ^{|\vx|}.
   \bigwedge\limits_{i \in [I]} e_i (\vx) = e'_i (\vx) \wedge
   \bigwedge\limits_{j \in [J]} f_j (\vx) \equiv f'_j (\vx) \mod n_j
   \implies
   h (\vx) \equiv h' (\vx) \mod m
 \end{equation*}
where $e_i (\vx), e'_i (\vx), f_j (\vx), f'_j (\vx), g (\vx), g'
(\vx), h (\vx), h' (\vx)
\in \bbfZ[\vx]$ for $i \in [I], j \in [J]$, and $m \in \bbfN$. 

It is not hard to rewrite modular polynomial equations in antecedents
of the above implications. For instance, the first implication is
equivalent to 
\[
\bbfZ \models \forall \vx. % \in \bbfZ^{|\vx|}.
\bigwedge\limits_{i \in [I]} e_i (\vx) = e'_i (\vx) \wedge
\bigwedge\limits_{j \in [J]} [\exists d_j. f_j (\vx) = f'_j (\vx) + d_j \cdot n_j]
\implies
g (\vx) = g' (\vx),
\]
which in turn is equivalent to
\[
\bbfZ \models \forall \vx \forall \vd. % \in \bbfZ^{|\vx|} \vd \in \bbfZ^{J}.
\bigwedge\limits_{i \in [I]} e_i (\vx) = e'_i (\vx) \wedge
\bigwedge\limits_{j \in [J]} f_j (\vx) = f'_j (\vx) + d_j \cdot n_j
\implies
g (\vx) = g' (\vx).
\]

It hence suffices to consider the following
\emph{polynomial equation entailment} problem:
\begin{equation}
  \label{eq:reduced-polynomial-equation-implicant}
  \bbfZ \models \forall \vx. % \in \bbfZ^{|\vx|}.
  \bigwedge\limits_{i \in [I]} e_i (\vx) = e'_i (\vx)
  \implies
  g (\vx) = g' (\vx); \textmd{ or}
\end{equation}
 \begin{equation}
   \label{eq:reduced-modular-polynomial-equation-implicant}
   \bbfZ \models \forall \vx. % \in \bbfZ^{|\vx|}.
   \bigwedge\limits_{i \in [I]} e_i (\vx) = e'_i (\vx)
   \implies
   h (\vx) \equiv h' (\vx) \mod m
 \end{equation}
 where $e_i (\vx), e'_i (\vx), g (\vx), g' (\vx), h (\vx), h' (\vx)
 \in \bbfZ[\vx]$ for $i \in [I]$ and $m \in \bbfN$~\cite{H:07:AENTP}.

We solve the polynomial equation entailment
problems~(\ref{eq:reduced-polynomial-equation-implicant}) 
and~(\ref{eq:reduced-modular-polynomial-equation-implicant}) via
the ideal membership problem~\cite{H:07:AENTP,BS:16:GFEV}. 
For~(\ref{eq:reduced-polynomial-equation-implicant}), consider the
ideal $I = \langle e_i(\vx) - e'_i(\vx) \rangle_{i \in [I]}$. Suppose
$g(\vx) - g'(\vx) \in I$. Then there are $u_i(\vx) \in \bbfZ[\vx]$
(called \emph{coefficients}) such that 
\begin{equation}
  \label{eq:reduced-polynomial-equation-witnesses}
  g(\vx) - g'(\vx) = \sum\limits_{i \in [I]} u_i (\vx) [e_i (\vx) - e'_i (\vx)].
\end{equation}
Hence $g(\vx) - g'(\vx) = 0$ follows from  the polynomial equations
$e_i (\vx) = e'_i (\vx)$ for $i \in [I]$. Similarly, it
suffices to check if $h(\vx) - h'(\vx) \in \langle m, e_i(\vx) -
e'_i(\vx) \rangle_{i \in [I]}$
for~(\ref{eq:reduced-modular-polynomial-equation-implicant}).
If so, there are $u, u_i(\vx) \in \bbfZ[\vx]$ such that
\begin{equation}
  \label{eq:reduced-modular-polynomial-equation-witnesses}
  h(\vx) - h'(\vx) = u(\vx) \cdot m + \sum\limits_{i \in [I]} u_i (\vx)
  [e_i (\vx) - e'_i (\vx)]. 
\end{equation}
Thus $h(\vx) \equiv h'(\vx) \mod m$ as required.
The reduction to the ideal membership problem however is 
incomplete. Consider $\bbfZ \models \forall x. x^2 + x
\equiv 0 \mod 2$ but $x^2 + x \not\in \langle 2
\rangle$~\cite{H:07:AENTP}. 

Two \coq tactics are available to
find formal proofs for the polynomial equation entailment
problems~\cite{P:08:CGBP,P:10:CGBP}. 
The tactic \dslcode{nsatz} proves the entailment problem of
the form in~(\ref{eq:reduced-polynomial-equation-implicant}); the
tactic \dslcode{gbarith} is able to prove the form
in~(\ref{eq:reduced-modular-polynomial-equation-implicant}). 
The ideal membership problem can be solved by finding a Gr\"obner
basis for the ideal. 
Both tactics solve the polynomial equation entailment problem by
computing Gr\"obner bases for induced ideals. Finding Gr\"obner bases
for ideals however is NP-hard because it allows us to solve a
system of equations over the Boolean field~\cite{GJ:1979:CAI}. 
Low-level mathematical constructs can have
hundreds of polynomial equations 
in~(\ref{eq:reduced-polynomial-equation-implicant})
or~(\ref{eq:reduced-modular-polynomial-equation-implicant}). 
Both \coq tactics fail to solve such problems in a reasonable
amount of time. 

We develop two heuristics to solve the polynomial equation
entailment problem more effectively. Note that the polynomial equations
generated by Algorithm~\ref{algorithm:polynomial-programs} are of the  
forms: $x = e$ (from assignment statements) or $x + 2^c y
= e$ (from \dslcode{Split} statements). Such polynomial equations can
safely be removed after every occurrences of $x$ are replaced with $e$
or $e - 2^c y$ respectively. The number of generators of the induced
ideal is hence reduced. We define a \coq
tactic to simplify polynomial equation entailment problems by 
rewriting variables and then removing polynomial equations.

To further improve scalability, we use the computer algebra system
\singular to solve the ideal membership problem~\cite{GP:08:SICA}. 
Our tactic submits the membership problem to \singular and 
obtains coefficients from the computer algebra system. 
Since algorithms used in \singular might be implemented incorrectly, 
our \coq
tactic then certifies the coefficients by checking the
equation~(\ref{eq:reduced-polynomial-equation-witnesses})
or~(\ref{eq:reduced-modular-polynomial-equation-witnesses}) to ensure
the polynomial equation entailment problem is correctly
solved. Soundness of our technique therefore does not rely on the
external solver \singular.

%%% Local Variables: 
%%% mode: latex
%%% eval: (TeX-PDF-mode 1)
%%% eval: (TeX-source-correlate-mode 1)
%%% TeX-master: "certified_vcg"
%%% End: 
